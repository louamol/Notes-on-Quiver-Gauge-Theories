\section{Properties of D-branes}

    \subsection{SYM from D-branes}

        The dynamics of D-branes is described by the Dirac-Born-Infeld action
        \begin{equation}
            S_{\text{DBI}}[X,F] = -\frac{T_p}{g_s}\int\d^{p+1}\sigma~\sqrt{-\det\limits_{0\leq a,b\leq p}(\eta_{ab}+\p_a X^m\p_b X_m+2\pi\alpha'F_{ab})}.
        \end{equation}
        The latter can be expended for slowly-varying fields, which is equivalent to passing to the field theory limit $\alpha'\to0$. The resulting action is the action of a $\U(1)$ gauge theory in $p+1$ dimensions with $9-p$ real scalar fields. This action is exactly the same than the one we would obtain by dimensionally-reducing a pure $\U(1)$ Yang-Mills gauge theory in 10 spacetime dimensions with the identification
        \begin{equation}
            g_{\text{YM}}=g_sT^{-1}_p(2\pi\alpha')^{-2}=\frac{g_s}{\sqrt{\alpha'}}(2\pi\sqrt{\alpha'})^{p-1}.
        \end{equation}

        This construction can be generalized for multiple D-branes. It now results in a non-abelian theory. The general statement is the following:
        \begin{result}
            The low-energy dynamics of $N$ parallel, coicident D$p$-branes in flat space is described in static gauge by the dimensional reduction to $p+1$ dimensions of pure $10d$ $\mN=1$ supersymmetric Yang-Mills theory with gauge group $\U(N)$ in ten spacetime dimensions.
        \end{result}
        Recall that the $10$-dimensional action is given by
        \begin{equation}
            S_{\text{YM}} = \frac{1}{4g^2_{\text{YM}}}\int\d^{10}x~\left[ \tr(F_{\mu\nu}F^{\mu\nu})+2i\tr(\bar{\psi}\Gamma^\mu D_\mu\psi)\right],\label{eq:SYMaction}
        \end{equation}
        where $F_{\mu\nu}=\p_\mu A_\nu-\p_\nu A_\mu+i[A_\mu,A_\nu]$ is the non-abelian field strength of the $\U(N)$ gauge field $A_\mu$, $D_\mu=\p_\mu-i[A_\mu,\psi]$, $\Gamma^\mu$ are $16\times 16$ Dirac matrices \todo{Why ?}, and the $N\times N$ Hermitian fermion field $\psi$ is a $16$-component Majorana-Weyl spinor of the Lorentz group $\SO(1,9)$ which transforms under the adjoint representation of the gauge group $\U(N)$. On-shell, there are eight on-shell bosonic, gauge field degrees of freedom, and eight fermionic degrees of freedom, after imposition of the Dirac equation $\xout{D}\psi=\Gamma^\mu D_\mu\psi=0$. One can verify that this action is invariant under the supersymmetry transformations
        \begin{align*}
            \delta_\eps A_\mu &= \frac{i}{2}\bar{\eps}\Gamma_\mu\psi,\\
            \delta_{\eps}\psi &= \frac{1}{2}F_{\mu\nu}[\Gamma^\mu,\Gamma^\nu]\eps,
        \end{align*}
        where $\eps$ is an Majorana-Weyl spinor.

        Using \eqref{eq:SYMaction}, we can construct a supersymmetric Yanf-Mills gauge theory in $p+1$ dimensions with $16$ independent supercharges by dimensional reduction: we take all fields to be independent of the coordinates $X^{p+1},\dots, X^9$, then the ten-dimensional gauge field $A_\mu$ splits into a $(p+1)$-dimensional $\U(N)$ gauge field $A_a$ plus $9-p$ Hermitian scalar fields $\Phi^m=X^m/2\pi\alpha'$ in the adjoint representation of $\U(N)$. The D$p$-brane action is thereby obtained from the dimensionality reduced field theory as
        \begin{equation}
            S_{\text{D}p} = -\frac{T_pg_s(2\pi\alpha')^2}{4}\int\d^{p+1}\sigma~\tr\left(F_{ab}F^{ab}+2D_a\Phi^m D^a\Phi_m+\sum_{m\neq n}[\Phi^m,\Phi^n]^2+\text{fermions}\right)\label{eq:SDp}
        \end{equation}
        where $a,b=0,\dots,p$, $m,n=p+1,\dots,9$. We do not explicitly display the fermionic contributions for the moment. In conclusion, the low-energy brane dynamics is described by a supersymmetric Yang-Mills theory on the D$p$-brane worldvolume which is dynamically coupled to the transverse, adjoint scalar fields $\Phi^m$.

        The scalar potential is given by
        \begin{equation}
            V(\Phi)=\sum_{m\neq n}[\Phi^m,\Phi^n]^2.
        \end{equation}
        It is negative definite because $[\Phi^m,\Phi^n]^\dagger=[\Phi^n,\Phi^m]=-[\Phi^m,\Phi^n]$. A classical vacuum of the field theory defined by \eqref{eq:SDp} corresponds to a static solution of the equations of motion whereby the potential energy of the system is minimized. It is given by the field configurations which solve simultaneously the quations $F_{ab}=D_a\Phi^m=\psi^a=0$ and $V(\Phi)=0$. Since all term in $V(\Phi)$ have the same sign, the equation $V(\Phi)=0$ is equivalent to the equation $[\Phi^m,\Phi^n]=0$ for all $m,n$ and at each point in the $(p+1)$-dimensional worldvolume of the branes. This implies that the $N\times N$ hermitian matrix fields $\Phi^m$ are simultaneously diagonalizable by a gauge transformation, so that we may write
        \begin{equation}
            \Phi^m=U
            \begin{bmatrix}
                X^m_1 & & & 0 \\
                & X^m_2 & & \\
                & & \ddots & \\
                0 & & & X^m_N
            \end{bmatrix}U^{-1},\label{eq:diagPhi}
        \end{equation}
        the matrix $U$ is independent of $m$. The simultaneous,
        real eigenvalues $X^m_i$ give the positions of the $N$ distinct D-branes in the $m$-th transverse direction. It follows that the moduli space of classical vacua for the $(p+1)$-dimensional field theory \eqref{eq:SDp} is the quotient space $(\R^{9-p})^N/S_N$, where the factors of $\R$ correspond to the positions of the $N$ D$p$-branes in the $(9-p)$-dimensional transverse space, and $S_N$ is the symmetric group acting by permutations of the $N$ coordinates $X_i$. The group $S_N$ corresponds to the residual Weyl symmetry of the $\U(N)$ gauge group acting in \eqref{eq:diagPhi}. It represents the permutation symmetry of a system of $N$ \emph{indistinguishable} D-branes.

        From \eqref{eq:SDp} one can easily deduce that the masses of the fields corresponding to the off-diagonal matrix elements are given precisely by the distances $\abs{x_i-x_j}$ between the corresponding branes. This description means that an interpretation of the D-brane configuration in terms of classical geometry is only possible in the classical ground state of the system, whereby the matrices $\Phi^m$ are simultaneously diagonalizable and the positions of the individual D-branes may be described through their spectrum of eigenvalues. This gives a simple and natural dynamical mechanism for the appearence of ``non-commutative geometry'' at short distances, where the D-branes cease to have well-defined positions according to classical geometry.

        \todo{The end of this section has to be rewritten}

    \subsection{D-branes and residual SUSY in type II theories}

        The minimal irreducible representation in 10 dimensions is a Majorana-Weyl representation of dimension 8. In type II theories, we have $\mN=(1,1)$ for IIA and $\mN=(2,0)$ for IIB. Because of the string origin of the generators, the two supersymmetry generators $\eps_L$ and $\eps_R$ (Majorana-Weyl spinors) satisfy
        \begin{equation}
            \eps_L=\Gamma_{11}\eps_L,\qquad \eps_R=\eta\Gamma_{11}\eps_R
        \end{equation}
        with $\eta=+1$ for IIB and $\eta=-1$ for IIA theory. For a D$p$-brane, the supersymmetry projections is the following:
        \begin{equation}
            \eps_L=\Gamma_0\dots\Gamma_p\eps_R.
        \end{equation}
        In other words, the supersymmetries with generators of the form
        \begin{equation}
            Q_\alpha+\Gamma_0\dots\Gamma_p\bar{Q}_{\dalpha}\label{eq:susypresved}
        \end{equation}
        are preserved by the D$p$-brane while the one with generators of the form
        \begin{equation}
            Q_\alpha-\Gamma_0\dots\Gamma_p\bar{Q}_{\dalpha}\label{eq:susybroken}
        \end{equation}
        are broken. They violate the boundary conditions. Since there is the same number of generators of the form \eqref{eq:susypresved} than of the form \eqref{eq:susybroken}, exactly haf of the supersymmetry is broken. The idea that one spacetime direction would break one supercharge could be reasonable if supersymmetries were transforming as vectors which not the case; supercharges transform as spinors. It would also be incompatible with the T-duality because two branes of different dimensions must have the same number of unbroken supercharges if there is a T-duality relating them: the number of unbroken supercharges is the same for all dual descriptions (a necessary condition for the equivalence). And indeed, in the correct theory, that's the case. Every type II D-brane breaks half of the supercharges.

        To obtain the previous relations, we start by the ones from M-theory and compactify the 11th direction, getting type IIA theory. $\Gamma_{11}$ then plays the role of the chiral projector in 10 dimensions; the supersymmetry parameters are related by $\eps_L=\frac{1}{2}(1+\Gamma_{11})\eps$ and $\eps_R=\frac{1}{2}(1-\Gamma_{11})\eps$. The relations for type IIB theory are then obtained by T-duality. Under a T-duality over the $\hat{i}$ direction, the supersymmetry parameters transform as
        \begin{align*}
            \eps_L &\mapsto \eps_L,\\
            \eps_R &\mapsto \Gamma_i\eps_R.
        \end{align*}
        The tension of a D$p$-brane is given by
        \begin{equation}
            T_{p} = \frac{1}{(2\pi)^pg_sl^{p+1}_s}.
        \end{equation}
        This completely fixes the Newton constant: the tension of electric-magnetic duals must satisfy:
        \begin{equation}
            T_pT_{D-p-4} = \frac{2\pi}{16\pi G_D}.
        \end{equation}
        In ten dimensions, this gives $G_{10}=8\pi^6g^2_sl^8_s$.

        The dualities are defined as follows:
        \begin{align*}
            \text{S-duality} &: g_s\mapsto\frac{1}{g_s},\qquad l^2_s\mapsto g_sl^2_s,\\
            \text{T-duality} &: R\mapsto\frac{l^2_s}{R},\qquad g\mapsto g_s\frac{l_s}{R}.
        \end{align*}

    \subsection{D-branes wrapping cycles}

        A D$p$-brane worldvolume $\phi:\Sigma\to X$ in spacetime $X$ \emph{wraps} a cycle $c\in H_{p+1}$ if the pushforward $\phi_*(\Sigma)\in H_\text{\textbullet}(X)$ of the fundamental class of $\Sigma$ is the class $[c]$ of the given cycle in $X$. If the pushforward is a mutliple of $[c]$, then the branes wraps $c$ multiple times.

\section{Algebraic geometry}

    \subsection{Elements}

        An important idea in algebraic geometry is that is is really the alegbra of function on it that defines a space. For affine varities $X$, this is illustrated by the fact that the structure of $X$ is really contained in its coordinate ring $K[x_1,\dots,x_n]/I(X)$ and by the isomorphism
        \begin{equation}
            K[x_1,\dots,x_n]|_X=K[x_1,\dots,x_n]/I(X).
        \end{equation}
        Now an algebraic set $Z(T)$ is irreducible if $I(Z(T))$ is prime. So there is a one-to-one correspondence between prime ideals and affine varieties.

        Given an algebraic variety, one can modify the equations continuously by varying some parameters and the variety will be ``deformed'' accordingly. It is called the \emph{variation of the omplex structure}. The space of all complex deformations of an affine variety $X$ is called the \emph{complex moduli space} of $X$. For a Calabi-Yau manifold, the linearization of the complex moduli space (tangent space) is given by the cohomology group $H^{m-1,1}(X)$, where $m$ is the dimension of $X$. In general, it is much more complicated.

    \subsection{Divisors and line bundles}

        A \emph{(Weyl) divisor} $D$ of a complex variety $X$ is a linear combination (formal sum with integer coefficients) of co-dimension one, irreducible subvarieties,
        \begin{equation}
            D=\sum_i n_iV_i,\qquad n_i\in\Z,V_i\subset X.
        \end{equation}
        It said to be effective if all $n_i\geq1$. To any line bundle $L$ with a regular section $s$ (that is, on any open subset $U_\alpha$, $s_\alpha=s|_{U_\alpha}$ is a polyomial in the local coordinates) we can associate a hypersurface $Y\subset X$ defined as
        \begin{equation}
            Y=\{p\in X|s(p)=0\}.
        \end{equation}
        This hypersurface $Y$ can then be decomposed into irreducible parts (affine patches) on which $s_\alpha$ can be factorized in $\C[x_1,\dots,x_n]$ and decomposed in prime ideals $P_i$ of multiplicity $n_i$. Assembling all the $V^\alpha_i$ otgether, we construct co-dimension one subvarities $V_i$ that can be used to form divisors. One can also proceed the other ay around and, given a divisor $D=\sum_i n_iV_i$, define a line bundle $\mathcal{O}_X(D)$ whose sections vanish on each $V_i$ with a zero of order $n_i$. This construction can be generalized to divisors with negative coefficients $n_i<0$ in which case now have poles of order $n_i$ in $V_i$.

    \subsection{Singularities and resolutions}

        A \emph{rational map} from a variety $X$ to another $Y$ is a morphism from a non-empty subset $U\subset X$ to $Y$. Recall that, by definition of the Zariski topology, a non-empty open subset is always dense. Concretely, a rational map can be written in coordinates using ration functions (quotient of polynomials). A \emph{birational map} is an invertible rational map. It induces an isomorphism between two non-empty open subsets. In this case, $X$ and $Y$ are said to be \emph{birationally equivalent}.

        The \emph{resolution of a singularity} of an algebraic variety $V$ is a non-singular variety $W$ with a proper birational map $W\to V$. For varities over fields of characteristic $0$, it was proven (Hironaka, 1964) that \todo{fill in}.

    \subsection{Projective plane curves}

        In $\C\P^2$, we consider a hypersurface defined by a single polynomial $p$ of degree $d$. If
        \begin{equation}
            \pdv{p(x)}{x_i}=0
        \end{equation}
        for all $i$ whenever $p(x)=0$, then the curve is said to be regular, it is a Riemann surface. The latter are classified by their genus and
        \begin{equation}
            g=\frac{(d-1)(d-2)}{2}.
        \end{equation}
        For $d=3$, the most geenral polynomial is
        \begin{equation}
            \sum_{i+j+k=3}c_{ijk}x^i_0x^j_1x^k_2=0
        \end{equation}
        and defines a torus, also called \emph{elliptic curve}. There $10$ independant parameters but $9$ of them can be removed by a $\GL(3,\C)$ transformation, leaving us with only one complex parameter; the complex structrue modulus of the torus.




    \todo{This section has to be rewritten.}

\section{Quivers in string theory and Yang-mills in graph theory}