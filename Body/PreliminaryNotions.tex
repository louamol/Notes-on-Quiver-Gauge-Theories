\part{Preliminary notions}

\section{Physical setup}

    \subsubsection*{Brane-world paradigm}

        We consider our four-dimensional world to be the worldvolume of a D$3$-brane in the ten-dimensional spacetime of type IIB superstring theory. More precisely, we consider a stack of $N$ D$3$-branes in order to have $\U(N)$ Chan-Paton factors resulting in a $\U(N)$ gauge group in the worldvolume theory. The spacetime is therefore not necessarily $\R^{1,9}$ but of the more general form
        \begin{equation*}
            M = \R^{1,3}\times M^{(6)}.\label{spacetimedecomp}
        \end{equation*}
        This is the so-called \emph{brane-world paradigm}. %In particular, we will be interested in type IIB string theory because of its self-duality under S-duality. \todo{Why ?}

    \subsubsection*{Supersymmetry and Calabi-Yau manifolds}
    
        Independently from string theory, we can ask for the wolrdvolume theory to be supersymmetric. We start from type IIB superstring theory which is $10$-dimensional and has $\mN=2$ supersymmetry so it possesses $32$ supercharges. As usual, they transform under the minimal spinor representation (MSR) of the bulk Lorentz group, here $\SO(1,9)$. In ten dimensions this representation is $8$-dimensional (complex) which is why there are $2(8+8)=32$ supercharges: $8$ transforming in the $8$-dimensional MSR and $8$ transforming in the $8$-dimensional conjugate MSR and the whole thing times two since $\mN=2$. Compactifying type II string theory on any $6$-dimensional manifold $M^{(6)}$ breaks supersymmetry. The reason for this is that the supercharges now have to transform under the MSR of $\SO(1,3)\times\H(M^{(6)})$\todo{Why ?}, where $\H(M^{(6)})$ is the holonomy group of $M^{6}$. Actually, the space $\R^{1,3}\times M^{(6)}$ can be viewed as the trivial bundle with base space $\R^{1,3}$ and fibers $M^{(6)}$, sincespinors takes values in the fibers they must transform under the holonomy group of $M^(6)$. A generic curved $6$-dimensional manifold has $\O(6)$ holonomy and $\SO(6)$ if it is orientable, as we will always consider. The supercharges must therefore transform under the MSR of $\SO(1,3)\times\SO(6)$. The MSR of $\SO(1,3)$ being $\boldsymbol{2}$ and the one of $\SO(6)$ being $\boldsymbol{4}$, we conclude that imposing the spacetime to have the form \eqref{spacetimedecomp} changes the representation under which the supercharges transform in the following way:
        \begin{equation}
            \boldsymbol{8}\oplus\bar{\boldsymbol{8}}\to (\boldsymbol{2}_L,\boldsymbol{4})\oplus(\boldsymbol{2}_R,\bar{\boldsymbol{4}}).
        \end{equation}
        If we stop here, the residual supercharges might be ill-defined; making a tour around a loop in $M^{(6)}$ could result in a non-trivial rotation. To solve this problem, we need to be more restrictive with the holonomy. In fact, it is precisely the holonomy of the transverse space that dictates the number of residual supersymmetries. To understand this, let us now consider a four-dimensional field theory resulting from compactification of the transverse six-dimensional space. The number of supercharges that generate supersymmetries for this theory is the number of Killing spinors (covariantly constant spinors) because each Killing spinor contracted with the local supersymmetry current generates a residual supersymmetry. Let us now make the link with holonomy: since $\SO(6)\cong\SU(4)$, minimal spinors can be viewed as having four complex components and as transforming under $\SU(4)$. Indeed, minimal spinors in six dimensions have four complex components. In order to have one covariantly conserved spinor, we look for the biggest subgroup of $\SU(4)$ that leaves a component of the spinor invariant. This is clearly $\{e\}\times\SU(3)\subset\SU(4)$ that acts trivially on the first component. The spinor $(1,0,0,0)$ is then covariantly constant. Our transverse space must therefore have $\SU(3)$ holonomy such that the parallel transport of the spinor $(1,0,0,0)$ under any closed loop is a lower $\SU(3)$ rotation. We conclude that if the transverse Calabi-Yau has $\SU(3)$ holonomy, the worldvolume theory has $\mN=1$ supersymmetry. If the holonomy is $\SU(2)\subset\SU(3)$, the spinor $(0,1,0,0)$ is also a Killing spinor which means that we have $\mN=2$ supersymmetry for the worldvolume theory. To summarize, preserving any degree of supersymmetry constrains the transverse space $M^{(6)}$ to be compact, complex, Kähler and to have $G\subset\SU(3)$ holonomy. Namely, $M^{(6)}$ must be a Calabi-Yau threefold, see section \ref{sec:CY}.

    \subsubsection*{Non-compact transverse space}
    
        If we let the worldvolume of the D$3$-branes carry the requisite gauge theory while the bulk contains gravity, we can relax the compactness condition and study non-compact Calabi-Yau threefolds. This makes the analysis much simpler and therefore also serves as an argument to ignore gravity in the worldvolume theory. Consequently, we will mostly ignore gravity and not care about the metric of the spacetime, see appendix \ref{app:spacetimegeom} for more details. In this setup we cannot really talk about compactification anymore. Instead we just think of it as a flat space on which lives the gauge theory while gravity only lives in transverse space. To understand intuitively why there is no gravity in this limit, we can think of Kaluza-Klein compactification. The four-dimensional gravity coupling constant is inversely proportional to the size of the compactifying space therefore there is no gravity in the non-compact/infinite-size limit. This more a motivation than a proof.

    \subsubsection*{Singular transverse space}

        The only non-compact smooth Calabi-Yau threefold is $\C^3$, this forces us to consider singular Calabi-Yau varieties is we want more interesting theories. A Calabi-Yau variety is an affine variety that locally models a Calabi-Yau manifold, therefore allowing for singularities. We usually denote $S\equiv M^{(6)}$ to remind us of the singular aspect. String theory being a theory of extended objects, turns out to be it is well-defined on such singularities and even ``smoothened'' the singularity in some sens. Considering strings on singular geometries requires to ``project'' the theory obtain from $\C^3$. As a result, the gauge group $\U(N)$ is broken down into products of smaller gauge groups. This ``projection'' highly depends on the type of singularity (orbifold, toric, del Pezzo, etc) we are considering. While it is relatively straightforward for ``simple'' singularities (e.g. abelian orbifolds) it quickly gets more complicated or even unknown for others.
    
        From the point of view of the orbifold, the D$3$-brane is a point, meaning that the D$3$-branes are really probing the transverse space and, in particular, they parametrize it. This is the first clue of the tight relationship that exists between the worldvolume theory and the transverse singular space. Eventually, we will see that the classical vacuum of the gauge theory should be, in explicit coordinates, the defining equation of $S$. This is precisely the opposite of the projection manipulation we mentioned above: recovering the transverse space from the gauge theory. Projecting and computing the classical vacua are therefore inverse operations with respect to each other. This suggest a bijection between the singular transverse space and the gauge theory: the former can be computed from the latter and vice-versa. This is called ``forward algorithm'' and ``inverse algorithm'' respectively.

    \subsubsection*{Mathematical formulation}

        Mathematically, this brane-world paradigm is the realization of branes as supports of vector bundles (sheaf). Gauge theories on branes are intimately related to algebraic constructions of stable bundles, i.e. holomorphic or algebraic vector bundles that are stable in the sense of geometric invariant theory. In particular, D-brane gauge theories manifest as a natural description of symplectic quotients and their resolutions in geometric invariant theory. Together with the stable vector bundle (sheaf) supported thereupon the D-branes resolve the transverse Calabi-Yau orbifold, which is the vacuum for the gauge theory on the worldvolume as a GIT quotient.

    \subsubsection*{Summary}

        We consider $N$ D$3$-branes in type IIB superstring theory carrying a $\U(N)$ gauge group. The transverse space $S$ is taken to be a non-compact singular Calabi-Yau variety.
    
\section{Properties of D-branes}

    \subsection{SYM from D-branes}

        The dynamics of D-branes is described by the Dirac-Born-Infeld action
        \begin{equation}
            S_{\text{DBI}}[X,F] = -\frac{T_p}{g_s}\int\d^{p+1}\sigma~\sqrt{-\det\limits_{0\leq a,b\leq p}(\eta_{ab}+\p_a X^m\p_b X_m+2\pi\alpha'F_{ab})}.
        \end{equation}
        The latter can be expended for slowly-varying fields, which is equivalent to passing to the field theory limit $\alpha'\to0$. The resulting action is the action of a $\U(1)$ gauge theory in $p+1$ dimensions with $9-p$ real scalar fields. This action is exactly the same than the one we would obtain by dimensionally-reducing a pure $\U(1)$ Yang-Mills gauge theory in 10 spacetime dimensions with the identification
        \begin{equation}
            g_{\text{YM}}=g_sT^{-1}_p(2\pi\alpha')^{-2}=\frac{g_s}{\sqrt{\alpha'}}(2\pi\sqrt{\alpha'})^{p-1}.
        \end{equation}

        This construction can be generalized for multiple D-branes. It now results in a non-abelian theory. The general statement is the following:
        \begin{result}
            The low-energy dynamics of $N$ parallel, coicident D$p$-branes in flat space is described in static gauge by the dimensional reduction to $p+1$ dimensions of pure $10d$ $\mN=1$ supersymmetric Yang-Mills theory with gauge group $\U(N)$ in ten spacetime dimensions.
        \end{result}
        Recall that the $10$-dimensional action is given by
        \begin{equation}
            S_{\text{YM}} = \frac{1}{4g^2_{\text{YM}}}\int\d^{10}x~\left[ \tr(F_{\mu\nu}F^{\mu\nu})+2i\tr(\bar{\psi}\Gamma^\mu D_\mu\psi)\right],\label{eq:SYMaction}
        \end{equation}
        where $F_{\mu\nu}=\p_\mu A_\nu-\p_\nu A_\mu+i[A_\mu,A_\nu]$ is the non-abelian field strength of the $\U(N)$ gauge field $A_\mu$, $D_\mu=\p_\mu-i[A_\mu,\psi]$, $\Gamma^\mu$ are $16\times 16$ Dirac matrices \todo{Why ?}, and the $N\times N$ Hermitian fermion field $\psi$ is a $16$-component Majorana-Weyl spinor of the Lorentz group $\SO(1,9)$ which transforms under the adjoint representation of the gauge group $\U(N)$. On-shell, there are eight on-shell bosonic, gauge field degrees of freedom, and eight fermionic degrees of freedom, after imposition of the Dirac equation $\xout{D}\psi=\Gamma^\mu D_\mu\psi=0$. One can verify that this action is invariant under the supersymmetry transformations
        \begin{align*}
            \delta_\eps A_\mu &= \frac{i}{2}\bar{\eps}\Gamma_\mu\psi,\\
            \delta_{\eps}\psi &= \frac{1}{2}F_{\mu\nu}[\Gamma^\mu,\Gamma^\nu]\eps,
        \end{align*}
        where $\eps$ is an Majorana-Weyl spinor.

        Using \eqref{eq:SYMaction}, we can construct a supersymmetric Yanf-Mills gauge theory in $p+1$ dimensions with $16$ independent supercharges by dimensional reduction: we take all fields to be independent of the coordinates $X^{p+1},\dots, X^9$, then the ten-dimensional gauge field $A_\mu$ splits into a $(p+1)$-dimensional $\U(N)$ gauge field $A_a$ plus $9-p$ Hermitian scalar fields $\Phi^m=X^m/2\pi\alpha'$ in the adjoint representation of $\U(N)$. The D$p$-brane action is thereby obtained from the dimensionality reduced field theory as
        \begin{equation}
            S_{\text{D}p} = -\frac{T_pg_s(2\pi\alpha')^2}{4}\int\d^{p+1}\sigma~\tr\left(F_{ab}F^{ab}+2D_a\Phi^m D^a\Phi_m+\sum_{m\neq n}[\Phi^m,\Phi^n]^2+\text{fermions}\right)\label{eq:SDp}
        \end{equation}
        where $a,b=0,\dots,p$, $m,n=p+1,\dots,9$. We do not explicitly display the fermionic contributions for the moment. In conclusion, the low-energy brane dynamics is described by a supersymmetric Yang-Mills theory on the D$p$-brane worldvolume which is dynamically coupled to the transverse, adjoint scalar fields $\Phi^m$.

        The scalar potential is given by
        \begin{equation}
            V(\Phi)=\sum_{m\neq n}[\Phi^m,\Phi^n]^2.
        \end{equation}
        It is negative definite because $[\Phi^m,\Phi^n]^\dagger=[\Phi^n,\Phi^m]=-[\Phi^m,\Phi^n]$. A classical vacuum of the field theory defined by \eqref{eq:SDp} corresponds to a static solution of the equations of motion whereby the potential energy of the system is minimized. It is given by the field configurations which solve simultaneously the quations $F_{ab}=D_a\Phi^m=\psi^a=0$ and $V(\Phi)=0$. Since all term in $V(\Phi)$ have the same sign, the equation $V(\Phi)=0$ is equivalent to the equation $[\Phi^m,\Phi^n]=0$ for all $m,n$ and at each point in the $(p+1)$-dimensional worldvolume of the branes. This implies that the $N\times N$ hermitian matrix fields $\Phi^m$ are simultaneously diagonalizable by a gauge transformation, so that we may write
        \begin{equation}
            \Phi^m=U
            \begin{bmatrix}
                X^m_1 & & & 0 \\
                & X^m_2 & & \\
                & & \ddots & \\
                0 & & & X^m_N
            \end{bmatrix}U^{-1},\label{eq:diagPhi}
        \end{equation}
        the matrix $U$ is independent of $m$. The simultaneous,
        real eigenvalues $X^m_i$ give the positions of the $N$ distinct D-branes in the $m$-th transverse direction. It follows that the moduli space of classical vacua for the $(p+1)$-dimensional field theory \eqref{eq:SDp} is the quotient space $(\R^{9-p})^N/S_N$, where the factors of $\R$ correspond to the positions of the $N$ D$p$-branes in the $(9-p)$-dimensional transverse space, and $S_N$ is the symmetric group acting by permutations of the $N$ coordinates $X_i$. The group $S_N$ corresponds to the residual Weyl symmetry of the $\U(N)$ gauge group acting in \eqref{eq:diagPhi}. It represents the permutation symmetry of a system of $N$ \emph{indistinguishable} D-branes.

        From \eqref{eq:SDp} one can easily deduce that the masses of the fields corresponding to the off-diagonal matrix elements are given precisely by the distances $\abs{x_i-x_j}$ between the corresponding branes. This description means that an interpretation of the D-brane configuration in terms of classical geometry is only possible in the classical ground state of the system, whereby the matrices $\Phi^m$ are simultaneously diagonalizable and the positions of the individual D-branes may be described through their spectrum of eigenvalues. This gives a simple and natural dynamical mechanism for the appearence of ``non-commutative geometry'' at short distances, where the D-branes cease to have well-defined positions according to classical geometry.

        \todo{The end of this section has to be rewritten}

    \subsection{D-branes and residual SUSY in type II theories}

        The minimal irreducible representation in 10 dimensions is a Majorana-Weyl representation of dimension 8. In type II theories, we have $\mN=(1,1)$ for IIA and $\mN=(2,0)$ for IIB. Because of the string origin of the generators, the two supersymmetry generators $\eps_L$ and $\eps_R$ (Majorana-Weyl spinors) satisfy
        \begin{equation}
            \eps_L=\Gamma_{11}\eps_L,\qquad \eps_R=\eta\Gamma_{11}\eps_R
        \end{equation}
        with $\eta=+1$ for IIB and $\eta=-1$ for IIA theory. For a D$p$-brane, the supersymmetry projections is the following:
        \begin{equation}
            \eps_L=\Gamma_0\dots\Gamma_p\eps_R.
        \end{equation}
        In other words, the supersymmetries with generators of the form
        \begin{equation}
            Q_\alpha+\Gamma_0\dots\Gamma_p\bar{Q}_{\dalpha}\label{eq:susypresved}
        \end{equation}
        are preserved by the D$p$-brane while the one with generators of the form
        \begin{equation}
            Q_\alpha-\Gamma_0\dots\Gamma_p\bar{Q}_{\dalpha}\label{eq:susybroken}
        \end{equation}
        are broken. They violate the boundary conditions. Since there is the same number of generators of the form \eqref{eq:susypresved} than of the form \eqref{eq:susybroken}, exactly haf of the supersymmetry is broken. The idea that one spacetime direction would break one supercharge could be reasonable if supersymmetries were transforming as vectors which not the case; supercharges transform as spinors. It would also be incompatible with the T-duality because two branes of different dimensions must have the same number of unbroken supercharges if there is a T-duality relating them: the number of unbroken supercharges is the same for all dual descriptions (a necessary condition for the equivalence). And indeed, in the correct theory, that's the case. Every type II D-brane breaks half of the supercharges.

        To obtain the previous relations, we start by the ones from M-theory and compactify the 11th direction, getting type IIA theory. $\Gamma_{11}$ then plays the role of the chiral projector in 10 dimensions; the supersymmetry parameters are related by $\eps_L=\frac{1}{2}(1+\Gamma_{11})\eps$ and $\eps_R=\frac{1}{2}(1-\Gamma_{11})\eps$. The relations for type IIB theory are then obtained by T-duality. Under a T-duality over the $\hat{i}$ direction, the supersymmetry parameters transform as
        \begin{align*}
            \eps_L &\mapsto \eps_L,\\
            \eps_R &\mapsto \Gamma_i\eps_R.
        \end{align*}
        The tension of a D$p$-brane is given by
        \begin{equation}
            T_{p} = \frac{1}{(2\pi)^pg_sl^{p+1}_s}.
        \end{equation}
        This completely fixes the Newton constant: the tension of electric-magnetic duals must satisfy:
        \begin{equation}
            T_pT_{D-p-4} = \frac{2\pi}{16\pi G_D}.
        \end{equation}
        In ten dimensions, this gives $G_{10}=8\pi^6g^2_sl^8_s$.

        The dualities are defined as follows:
        \begin{align*}
            \text{S-duality} &: g_s\mapsto\frac{1}{g_s},\qquad l^2_s\mapsto g_sl^2_s,\\
            \text{T-duality} &: R\mapsto\frac{l^2_s}{R},\qquad g\mapsto g_s\frac{l_s}{R}.
        \end{align*}

    \subsection{D-branes wrapping cycles}

        A D$p$-brane worldvolume $\phi:\Sigma\to X$ in spacetime $X$ \emph{wraps} a cycle $c\in H_{p+1}$ if the pushforward $\phi_*(\Sigma)\in H_\text{\textbullet}(X)$ of the fundamental class of $\Sigma$ is the class $[c]$ of the given cycle in $X$. If the pushforward is a mutliple of $[c]$, then the branes wraps $c$ multiple times.

\section{Algebraic geometry}

    \subsection{Elements}

        An important idea in algebraic geometry is that is is really the alegbra of function on it that defines a space. For affine varities $X$, this is illustrated by the fact that the structure of $X$ is really contained in its coordinate ring $K[x_1,\dots,x_n]/I(X)$ and by the isomorphism
        \begin{equation}
            K[x_1,\dots,x_n]|_X=K[x_1,\dots,x_n]/I(X).
        \end{equation}
        Now an algebraic set $Z(T)$ is irreducible if $I(Z(T))$ is prime. So there is a one-to-one correspondence between prime ideals and affine varieties.

        Given an algebraic variety, one can modify the equations continuously by varying some parameters and the variety will be ``deformed'' accordingly. It is called the \emph{variation of the omplex structure}. The space of all complex deformations of an affine variety $X$ is called the \emph{complex moduli space} of $X$. For a Calabi-Yau manifold, the linearization of the complex moduli space (tangent space) is given by the cohomology group $H^{m-1,1}(X)$, where $m$ is the dimension of $X$. In general, it is much more complicated.

    \subsection{Divisors and line bundles}

        A \emph{(Weyl) divisor} $D$ of a complex variety $X$ is a linear combination (formal sum with integer coefficients) of co-dimension one, irreducible subvarieties,
        \begin{equation}
            D=\sum_i n_iV_i,\qquad n_i\in\Z,V_i\subset X.
        \end{equation}
        It said to be effective if all $n_i\geq1$. To any line bundle $L$ with a regular section $s$ (that is, on any open subset $U_\alpha$, $s_\alpha=s|_{U_\alpha}$ is a polyomial in the local coordinates) we can associate a hypersurface $Y\subset X$ defined as
        \begin{equation}
            Y=\{p\in X|s(p)=0\}.
        \end{equation}
        This hypersurface $Y$ can then be decomposed into irreducible parts (affine patches) on which $s_\alpha$ can be factorized in $\C[x_1,\dots,x_n]$ and decomposed in prime ideals $P_i$ of multiplicity $n_i$. Assembling all the $V^\alpha_i$ otgether, we construct co-dimension one subvarities $V_i$ that can be used to form divisors. One can also proceed the other ay around and, given a divisor $D=\sum_i n_iV_i$, define a line bundle $\mathcal{O}_X(D)$ whose sections vanish on each $V_i$ with a zero of order $n_i$. This construction can be generalized to divisors with negative coefficients $n_i<0$ in which case now have poles of order $n_i$ in $V_i$.

    \subsection{Singularities and resolutions}

        A \emph{rational map} from a variety $X$ to another $Y$ is a morphism from a non-empty subset $U\subset X$ to $Y$. Recall that, by definition of the Zariski topology, a non-empty open subset is always dense. Concretely, a rational map can be written in coordinates using ration functions (quotient of polynomials). A \emph{birational map} is an invertible rational map. It induces an isomorphism between two non-empty open subsets. In this case, $X$ and $Y$ are said to be \emph{birationally equivalent}.

        The \emph{resolution of a singularity} of an algebraic variety $V$ is a non-singular variety $W$ with a proper birational map $W\to V$. For varities over fields of characteristic $0$, it was proven (Hironaka, 1964) that \todo{fill in}.

    \subsection{Projective plane curves}

        In $\C\P^2$, we consider a hypersurface defined by a single polynomial $p$ of degree $d$. If
        \begin{equation}
            \pdv{p(x)}{x_i}=0
        \end{equation}
        for all $i$ whenever $p(x)=0$, then the curve is said to be regular, it is a Riemann surface. The latter are classified by their genus and
        \begin{equation}
            g=\frac{(d-1)(d-2)}{2}.
        \end{equation}
        For $d=3$, the most geenral polynomial is
        \begin{equation}
            \sum_{i+j+k=3}c_{ijk}x^i_0x^j_1x^k_2=0
        \end{equation}
        and defines a torus, also called \emph{elliptic curve}. There $10$ independant parameters but $9$ of them can be removed by a $\GL(3,\C)$ transformation, leaving us with only one complex parameter; the complex structrue modulus of the torus.




    \todo{This section has to be rewritten.}

\section{Quivers in string theory and Yang-mills in graph theory}