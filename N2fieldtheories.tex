\part{Non-singular case}

\section{$\mN=2$ $\SU(N_c)$ supersymmetric field theories}

    We consider an $\mN=2$ gauge theory with gauge group $\SU(N_c)$ and with $N_f$ hypermultiplets, i.e. $\mN=2$ SQCD with $N_c$ colors and $N_f$ flavors. Recall the following decomposition of $\mN=2$ superfields in terms of $\mN=1$ superfields:
    \begin{align}
        [\mN=2 \text{ vector multiplet}] &: V=(\lambda_\alpha,A_\mu,D)\oplus \Phi=(\phi,\psi_\alpha,F)\\
        [\mN=2 \text{ hypermultiplet}] &: Q=(H_1,\psi_{1\alpha},F_1) \oplus \tilde{Q}=(\bar{H}_2,\bar{\psi}_{2\dalpha},\bar{F}_2)
    \end{align}
    where $V$ is a vector superfield and $\Phi,H_1,H_2$ are chiral superfields. We denote by $\W_\alpha$ the chiral superfield strength associated to $V$. We have
    \begin{itemize}
        \item $V$ is a vector superfield transforming in the adjoint of $\SU(N_c)$. It belongs to $\mathfrak{su}(N_c)$ and his components are denoted by $V^a_b$ with $a,b=1,\dots,N_c$.
        \item $\Phi$ is a chiral superfield transforming in the adjoint of $\SU(N_c)$. It belongs to $\mathfrak{su}(N_c)$ and his components are denoted by $\Phi^a_b$ with $a,b=1,\dots,N_c$.
        \item $Q^i$ ($i=1,\dots,N_f$) are $N_f$ chiral superfields transforming in the $\boldsymbol{N_C}$ of $\SU(N_c)$ and in the $\boldsymbol{N_f}$ of the global group $\SU(N_f)$. It has $N_c$ components, denoted by $Q^i_a$.
        \item $\tilde{Q}_i$ are $N_f$ chiral superfields transforming in the $\bar{\boldsymbol{N_C}}$ of $\SU(N_c)$ and in the $\bar{\boldsymbol{N_f}}$ of the global group $\SU(N_f)$. It has $N_c$ components, denoted by $\tilde{Q}^a_i$.
    \end{itemize}
    The lagrangian reads
    \begin{equation}
        \L^{\mN=2}_{\text{SYM}} = \frac{1}{4\pi}\Im\left[\tau\int\d^2\theta\d^2\bar{\theta}\tr\left(\Phi^\dagger e^V\Phi + Q^\dagger_i e^V Q^i + \tilde{Q}^{\dagger i}e^V \tilde{Q}_i\right)+\tau\int\d^2\theta\left(\frac{1}{2}\tr(\W^\alpha\W_\alpha)+W(\phi,H_1,H_2)\right)\right]\label{eq:lag}
    \end{equation}
    where $W(H_1,H_2)$ is the $\mN=2$ superpotential
    \begin{align}
        W(\phi,H_1,H_2) &= \sqrt{2}H_1\phi H_2 + mH_1H_2 \\
        &= \sqrt{2}(H_2)^a_i\phi^b_a (H_1)^i_b + \sqrt{2}m^i_j(H_2)^a_i(H_1)^j_a
    \end{align}
    and $\tau$ is the complexified gauge coupling
    \begin{equation}
        \tau=\frac{\theta}{\pi}+i\frac{8\pi}{g^2}.
    \end{equation}
    The matrix $m$ has to satisfy
    \begin{equation}
        [m,m^\dagger]=0
    \end{equation}
    in order to preserve $\mN=2$ supersymmetry, it is called the \emph{quark mass matrix}. This matrix can be diagonalized by an $\SU(N_f)$ transformation, i.e. a flavor rotation, to become
    \begin{equation}
        m=\text{diag}(m_1,\dots,m_{N_f}).
    \end{equation}

    Classically and with $m=0$ the global symmetry should be $\SU(N_f)\times\U(1)_B\times\U(2)_R$. The mass terms and instanton corrections breaks $\U(1)_R$ of the $R$-symmetry, leaving the compact component $\SU(2)_R$ unbroken. The lagrangian should be invariant under the latter, it is a necessary and sufficient condition to have $\mN=2$ supersymmetry. Under the unbroken $\SU(2)_R$, the bosonic fields of the vector multiplet, i.e. $A_\mu,\phi,D,F$ are singlets butthe fermions form a doublet $(\lambda_\alpha,\psi_\alpha)$. Similarly, for the hypermultiplets, the fermions $\psi_{1\alpha},\bar{\psi}_{2\dalpha}$ are singlets while their scalar superpartners for a doublet $(H_1,\bar{H_2})$. The $\SU(2)_R$ symmetry cannot be made manifest in terms of $\mN=1$ sueprfields but the symmetry $\U(1)_J\subset\SU(2)_R$ is manifest in \eqref{eq:lag}.
    
    The selection rules resulting from the breaking of the classical symmetries by mass terms and instanton corrections can be describe bt assigning symmetry transformation properties to the corresponding parameters in the action. In particular, the quark mass matrix $m$ can be decomposed into a trace part $m_S$ that transforms as a singlet under $\SU(N_f)$ and a traceless part $m_A$ that transforms in the adjoint of $\SU(N_f)$. We summarize all the representations in which the fields and the parameters transform transform in table \ref{table:fieldrepr}.

    \begin{table}[H]
        \centering
        $
        \begin{array}{c|ccccc}
            & \SU(N_c) & \SU(N_f) & \U(1)_B & \U(1)_R & \U(1)_J \\ \hline
            \Phi & \textbf{adj} & \boldsymbol{1} & 0 & 2 & 0 \\
            Q & \boldsymbol{N_c} & \boldsymbol{N_f} & 1 & 0 & 1 \\
            \tilde{Q} & \bar{\boldsymbol{N_c}} & \bar{\boldsymbol{N_f}} & -1 & 0 & 1 \\
            m_A & \boldsymbol{1} & \textbf{adj} & 0 & 2 & 0 \\
            m_S & \boldsymbol{1} & \boldsymbol{1} & 0 & 2 & 0 \\
            \Lambda^{2N_c-N_f} & \boldsymbol{1} & \boldsymbol{1} & 0 & 2(2N_c-N_f) & 0
        \end{array}
        $
        \caption{Field representations.}
        \label{table:fieldrepr}
    \end{table}

    For $\mN=2$ theories, the $\beta$ function is exact at $1$-loop and $\beta_{1\text{-loop}}\propto 2N_c-N_f$. If if $N_f<2N_c$, the $\beta$-function is negative. The theory is asymptotically free and it generates a strong-coupling scale $\Lambda$. The instanton factor is proportional to $\Lambda^{2N_c-N_f}$ and the $\U(1)_R$ symmetry is anomalous. It is broken down to a discrete $\Z_{2N_f-N_c}$ symmetry. For $N_f=2N_c$, the theory is scale invariant and $\U(1)_R$ symmetry is not anomalous. No strong-coupling scale is generated and the theory is described in terms of its bare couplings.

    $D,F,F_1$ and $F_2$ are auxiliary fields and their equations of motion are:
    \begin{align}
        F^a_b &= \pdv{W}{\phi^b_a} = \sqrt{2}(H_2)^a_i (H_1)^i_b\\
        (F_1)^a_i &= \pdv{W}{(H_{1})^i_a} = \sqrt{2}(H_2)^b_i\phi^a_b + \sqrt{2}m^j_i(H_2)^a_j,\\
        (F_2)^i_a &= \pdv{W}{(H_{2})^a_i} = \sqrt{2}\phi^b_a (H_1)^i_b + \sqrt{2}m^i_j(H_1)^j_a,\\
        D^A &= -[\phi,\phi^\dagger]^A + \bar{H}_1T^AH_1-\bar{H}_2T^AH_2
    \end{align}
    where $T^A$ are the generators of $\SU(N_f)$ and $A=1,\dots,N^2_f-1$. Note that we can also integrate out the auxiliary fields $F_1$ and $F_2$ to recast the scalar potential for the hypermultiplets as a D-term contribution. The potential reads
    \begin{align}
        V(\phi,H_1,H_2) &= \frac{1}{2}\tr(D^AD_A)+\bar{F}F+\bar{F_1}F_1+\bar{F_2}F_2\\
        &= \frac{1}{2}\tr([\phi,\phi^\dagger]^2)+\frac{1}{2}\abs{\bar{H}_1T^AH_1-\bar{H}_2T^AH_2}^2\\
        &\qquad+2\abs{(H_2)^b_i\phi^a_b+m^j_i(H_2)^a_j}^2+2\abs{\phi^b_a (H_1)^i_b+m^i_j(H_1)^j_a}^2
    \end{align}

\section{Classical moduli space}

    The D-term equations are
    \begin{align}
        D:
        \begin{cases}
            \hspace{3.1cm}[\phi,\phi^\dagger]  &= 0 \\
            (H_1)^i_a(H^\dagger_1)^b_i-(H^\dagger_2)^i_a(H_2)^b_i &= \nu\delta^a_b
        \end{cases}
    \end{align}
    and the F-term equations are
    \begin{align}
        F:
        \begin{cases}
            \hspace{1.3cm}(H_1)^i_a(H_2)^b_i &= \rho\delta^b_a \\
            (H_1)^j_am^i_j+\phi^b_a(H_1)^i_b &= 0 \\
            m^i_j(H_2)^a_j+(H_2)^b_i\phi^a_b &= 0
        \end{cases}
    \end{align}\todo{Verify how to obtain these equation from the F-terms and D-terms}
    where $\nu$ and $\rho$ are arbitrary complex numbers.\todo{From where do those come from ?} The the two equations in the D-terms appear separately is a consequence of $\mN=2$ supersymmetry. One can square the D-term and show that the cross-term cancels or by noting that the first term is an $\SU(2)_R$-singlet and that that the second is part of a triplet\footnote{More generally, we will need to quotient by the complexified gauge transformation, which can be used to diagonalize $\phi$ and the first equation is automatically satisfied. This is another explanation.}.
    
    These equations suggest that $\phi,H_1$ and $H_2$ may get VEVs, which we denote by $\vev{\phi},\vev{H_1}$ and $\vev{H_2}$ respectively. Since there $N^2_c-1$ components $\phi^a_b$, $N_c\cdot N_f$ components $(H_1)^i_a$ and $N_c\cdot N_f$ components $(H_2)^a_i$, there are $N_c(N_c+2N_f)-1$ complex scalars in total. Meaning that the D-term and F-term equations define a subspace of $\C^{N_c(N_c+2N_f)-1}$. The \emph{classical moduli space} is defined as
    \begin{equation}
        \M_c\equiv Z(F,D)/G\subset \C^{N_c(N_c+2N_f)-1}
    \end{equation}
    where $G=\SU(N_c)$ is the gauge group. It turns out that we can just consider the F-term equations if we quotient by the complexified gauge group:
    \begin{equation}
        \M_c = Z(F)/G_\C.
    \end{equation}

    The solutions to those equations fall into various branches corresponding to the phases of the theory. The \emph{Coulomb branch} is the region of the moduli space where only the scalars from the vector multiplet can take VEVs, i.e. where $\vev{H_1}=\vev{H_2}=0$. The \emph{Higgs branch} is the region of the moduli space where only the scalars from the hypermultiplets can take VEVs, i.e. where $\vev{\phi}=0$. \emph{Mixed branches} are regions where all VEVs are non-vanishing. For simplicity we will mostly consider the case with no mass: $m^i_j=0$.

    \subsection{Coulomb branch}

        The only non-trivial equation is the first D-term equation $[\phi,\phi^\dagger]=0$, the other four are automatically satisfied. This equation is if and only $\phi$ belongs to $\mathfrak{h}_\C$, the complexified Cartan subalgebra of $\mathfrak{su}(N_c)$. In our case, this means that the scalar fields matrix $\phi$ can be diagonalized using a color rotation and put in the form
        \begin{equation}
            \phi = \sum_I \phi_Ih^I
        \end{equation}
        where $h^I=E_{I,I}-E_{I+1,I+1}$ with $(E_{I,J})_{ab}=\delta_{aI}\delta{bJ}\equiv$ are the generators of the Cartan subalgebra and $I=1,\dots N_c-1$ ($N_c-1$ is the rank of $\mathfrak{su}(N_c)$). In simpler words, the vacuum configurations are of the form
        \begin{equation}
            \phi=\text{diag}(\phi_1,\dots,\phi_{N_c}),\qquad \sum^{N_c}_{a=1}\phi_a=0.\label{eq:diagform}
        \end{equation}
        The vacuum configurations then depend on $N_c-1$ complex numbers so the Coulomb branch is a quotient of $\C^{N_c-1}$.
        
        At a generic point, the gauge group is broken to $\U(1)^r\times W$, where $W_G$ is the Weyl group of the gauge group, the group of residual gauge symmetries, while acting on $\phi$, do not not take it out of the Cartan subalgebra, i.e. keeps it the form \eqref{eq:diagform}. The low energy dynamic is the that of $r$ massless vector multiplets and $\dim G-r$ massive ones, with masses depending on the specific VEV's. The Weyl group of $\SU(N_c)$ is $S_{N_c-1}$. At last, the classical Coulomb branch is
        \begin{equation}
            \boxed{\M^V_c=\frac{\C^{N_c-1}}{S_{N_c-1}}.}
        \end{equation}
        A natural set of $\U(1)^{N-1}\times S_{N-1}$ invariant coordinates on this $(N_c-1)$-dimensional Coulomb branch can be shown to be
        \begin{equation}
            u_2=\sum_{i<j}\phi_i\phi_j,\quad u_3=\sum_{i<j<k}\phi_i\phi_j\phi_k,\quad \dots,\quad u_{N_c}=\phi_1\dots \phi_{N_c}, \qquad i,j,k=1,\dots,N_c.
        \end{equation}
        It has an orbifold singularity along submanifolds where some of the $\phi_a$'s are equal. In this case, some of the non-abelian gauge symmetry is restored. The scalar potential gives the mass of the fields $H_1$ and $H_2$ as $\phi_a+m_i$. The vanishing of these masses describes a complex co-dimension $1$ submanifold of the Coulomb branch. 

    \subsection{Higgs branch}

        Since we consider a vanishing quark mass matrix, only the second D-term equation and the first F-term equation are non-trivial. Recall that the squark fields $H_1$ and $H_2$ are complex matrices of size $N_c\times N_f$ and $N_f\times N_c$ respectively:
        \begin{equation}
            H_1=
            \begin{bmatrix}
                (H_1)^1_1 & \dots & (H_1)^{N_f} \\
                \vdots & & \vdots \\
                (H_1)^1_{N_c} & \dots & (H_1)^{N_f}_{N_c}
            \end{bmatrix},\qquad
            (H_2)^t=
            \begin{bmatrix}
                (H_2)^1_1 & \dots & (H_2)^{N_f} \\
                \vdots & & \vdots \\
                (H_2)^1_{N_c} & \dots & (H_2)^{N_f}_{N_c}
            \end{bmatrix}.
        \end{equation}

        \subsubsection{Squark VEV solutions}

            \begin{itemize}
                \item \underline{$N_f\geq2N_c$:} any solution can be put using flavor and color rotations:
                \begin{align}
                    \begin{split}
                    H_1 &= 
                    \begin{bmatrix}
                        \kappa_1 & & & 0 & & & 0 & \\
                        & \ddots & & & \ddots & & & \ddots \\
                        & & \kappa_{N_c} & & & 0 & & 
                    \end{bmatrix},\\
                    (H_2)^t &= 
                    \begin{bmatrix}
                        \tilde{\kappa}_1 & & & \lambda_1 & & & 0 & \\
                        & \ddots & & & \ddots & & & \ddots \\
                        & & \tilde{\kappa}_{N_c} & & & \lambda_{N_c} & & 
                    \end{bmatrix}
                \end{split}\label{eq:Higgsbranchsol}
                \end{align}
                where
                \begin{align}
                    \kappa_a\tilde{\kappa}_a &= \rho,\qquad\rho\in\C \label{eq:Higgsbrachcdt1}\\
                    \lambda^2_a &= \kappa^2_a-\frac{\abs{\rho}^2}{\kappa^2_a}+\nu,\qquad\nu\in\R \label{eq:Higgsbrachcdt2}
                \end{align}
                and the $\kappa_a's$ are non-zero if $\rho$ is non-zero.
                \item \underline{$N_f<2N_c$:} starting from a solution for $N_f=2N_c$ with some vanishing flavor columns, one can always construct a solution for $N_f<2N_c$ by removing those columns. On the other hand, starting from a solution for $N_f<2N_c$, one can always add vanishing flavor columns th construct a solution for $N_f=2N_c$. The necessary flavor rotation to put the solution into the form \eqref{eq:Higgsbranchsol} can be chosen not to act on these extra columns of zeros. This ensures us that this column-reduction procedure from $N_f=2N_c$ solutions will generate an $N_f<2N_c$ solution in every flavor orbit.
                
                To reduce \eqref{eq:Higgsbranchsol} by $2N_c-N_f$ columns, we must set $2N_c-N_f$ parameters to zero: $\lambda_1=\dots=\lambda_i=\kappa_1=\dots=\kappa_j=0$ with $i+j=2N_c-N_f$. By \eqref{eq:Higgsbrachcdt1}-\eqref{eq:Higgsbrachcdt2}, if some $\kappa$'s vanish, we must set $\rho=0$ before, which implies that some $\lambda_a$'s vanish too. Consequently, there are two possibilities to reducing columns, hence defining two sub-branches of the Higgs branch:
                \begin{itemize}[label=$\triangleright$]
                    \item \emph{baryonic branch}: only some $\lambda_a$'s vanish, more precisely, $i=2N_c-N_f$ of them and the VEV's have the form
                    \begin{align}
                        \begin{split}
                        H_1 &= 
                        \begin{bmatrix}
                            \kappa_1 & & & & & & \phantom{\lambda_1} & & \\
                            & \ddots & & & & & & \phantom{\ddots} & \\
                            & & \kappa_{N_f-N_c} & & & & & & \phantom{\lambda_{N_f-N_c}} \\
                            & & & \kappa_0 & & & & & \\
                            & & & & \ddots & & & & \\
                            & & & & & \kappa_0 & & &
                        \end{bmatrix},\qquad\kappa_a\in\R^+\\
                        (H_2)^t &= 
                        \begin{bmatrix}
                            \kappa_1 & & & & & & \lambda_1 & & \\
                            & \ddots & & & & & & \ddots & \\
                            & & \kappa_{N_f-N_c} & & & & & & \lambda_{N_f-N_c} \\
                            & & & \tilde{\kappa}_0 & & & & & \\
                            & & & & \ddots & & & & \\
                            & & & & & \tilde{\kappa}_0 & & &
                        \end{bmatrix},\qquad\lambda_a\in\R^+\\
                    \end{split}
                    \end{align}
                    where
                    \begin{align}
                        \kappa_a\tilde{\kappa}_a &= \rho,\qquad\rho\in\C \label{eq:Higgsbrachcdt1}\\
                        \lambda^2_a &= \kappa^2_a-\kappa^2_0+\abs{\rho}^2\left(\frac{1}{\kappa^2_a}-\frac{1}{\kappa^2_0}\right),\qquad\nu\in\R \label{eq:Higgsbrachcdt2}
                    \end{align}
                    We use the term baryonic branch for the $N_f\geq 2N_c$ solutions \eqref{eq:Higgsbranchsol} as well. The baryonic branch exists for $N_f\geq N_c$.
                    \item \emph{non-baryonic branch}:
                \end{itemize}
                
                There two possibilities, either only $\lambda$'s vanish
            \end{itemize}
        
        \subsubsection{Gauge symmetry and separate branches}

        \subsubsection{Flavor symmetry}

        \subsubsection{Gaue-invariant description}

    \subsection{Mixed branches}

\section{Quantum moduli space}

\part{$A_1$ singularity}

\part{$A_n$ singularity}

\part{$\D_4$ singularity}