%---PACKAGES----------------------------------------
\documentclass[a4paper,10pt]{article}

\usepackage{import}
\import{Packages/}{custom_packages.tex}
\import{Packages/}{custom_macros.tex}

\title{\textbf{Notes on Quiver Gauge Theories}}
\author{Louan Mol}
\date{Last updated on \today.}

% DOCUMENT -----------------------------

\begin{document}

%
\begin{titlepage}

	%\begin{tikzpicture}[remember picture,overlay] \node[opacity=0.15,inner sep=0pt] at (current page.center){\includegraphics[width=\paperwidth,height=\paperheight]{Pictures/wave_wall_paper2.png}};
    %\end{tikzpicture}

	\begin{center}

	{\Huge{\bfseries{Notes on Quiver Gauge Theories}}}\\[0.7cm]

	Louan Mol - \textit{Université Libre de Bruxelles}

    \vspace{1cm}

	\begin{figure}[H]
        \centering
        \includegraphics[scale=0.4]{Pictures/E8_graph.png}
    \end{figure}

    \vspace{2cm}
	
	{\large\textbf{Abstract}}
	\end{center}
	
	    \quad In these notes, we present some basic ideas around the large topic of quiver gauge theories, more precisely about their brane probes construction.  The goal is to reproduce and regroup the basics of these theories for various types of singularities, with increasing level of complexity (orbifold, toric, del Pezzo, etc). Note that this document is only meant as a work support and contains a lot of typos, errors and imprecisions.
	    
	\vfill

	Last update on \today.
	
\end{titlepage}

\maketitle

\tableofcontents

\pagebreak

\nocite{*}

\part{Non-singular case}

\section{$\mN=2$ $\SU(N_c)$ supersymmetric field theories}

    We consider an $\mN=2$ gauge theory with gauge group $\SU(N_c)$ and with $N_f$ hypermultiplets, i.e. $\mN=2$ SQCD with $N_c$ colors and $N_f$ flavors. Recall the following decomposition of $\mN=2$ superfields in terms of $\mN=1$ superfields:
    \begin{align}
        [\mN=2 \text{ vector multiplet}] &: V=(\lambda_\alpha,A_\mu,D)\oplus \Phi=(\phi,\psi_\alpha,F)\\
        [\mN=2 \text{ hypermultiplet}] &: Q=(H_1,\psi_{1\alpha},F_1) \oplus \tilde{Q}=(\bar{H}_2,\bar{\psi}_{2\dalpha},\bar{F}_2)
    \end{align}
    where $V$ is a vector superfield and $\Phi,H_1,H_2$ are chiral superfields. We denote by $\W_\alpha$ the chiral superfield strength associated to $V$. We have
    \begin{itemize}
        \item $V$ is a vector superfield transforming in the adjoint of $\SU(N_c)$. It belongs to $\mathfrak{su}(N_c)$ and his components are denoted by $V^a_b$ with $a,b=1,\dots,N_c$.
        \item $\Phi$ is a chiral superfield transforming in the adjoint of $\SU(N_c)$. It belongs to $\mathfrak{su}(N_c)$ and his components are denoted by $\Phi^a_b$ with $a,b=1,\dots,N_c$.
        \item $Q^i$ ($i=1,\dots,N_f$) are $N_f$ chiral superfields transforming in the $\boldsymbol{N_C}$ of $\SU(N_c)$ and in the $\boldsymbol{N_f}$ of the global group $\SU(N_f)$. It has $N_c$ components, denoted by $Q^i_a$.
        \item $\tilde{Q}_i$ are $N_f$ chiral superfields transforming in the $\bar{\boldsymbol{N_C}}$ of $\SU(N_c)$ and in the $\bar{\boldsymbol{N_f}}$ of the global group $\SU(N_f)$. It has $N_c$ components, denoted by $\tilde{Q}^a_i$.
    \end{itemize}
    The lagrangian reads
    \begin{equation}
        \L^{\mN=2}_{\text{SYM}} = \frac{1}{4\pi}\Im\left[\tau\int\d^2\theta\d^2\bar{\theta}\tr\left(\Phi^\dagger e^V\Phi + Q^\dagger_i e^V Q^i + \tilde{Q}^{\dagger i}e^V \tilde{Q}_i\right)+\tau\int\d^2\theta\left(\frac{1}{2}\tr(\W^\alpha\W_\alpha)+W(\phi,H_1,H_2)\right)\right]\label{eq:lag}
    \end{equation}
    where $W(H_1,H_2)$ is the $\mN=2$ superpotential
    \begin{align}
        W(\phi,H_1,H_2) &= \sqrt{2}H_1\phi H_2 + mH_1H_2 \\
        &= \sqrt{2}(H_2)^a_i\phi^b_a (H_1)^i_b + \sqrt{2}m^i_j(H_2)^a_i(H_1)^j_a
    \end{align}
    and $\tau$ is the complexified gauge coupling
    \begin{equation}
        \tau=\frac{\theta}{\pi}+i\frac{8\pi}{g^2}.
    \end{equation}
    The matrix $m$ has to satisfy
    \begin{equation}
        [m,m^\dagger]=0
    \end{equation}
    in order to preserve $\mN=2$ supersymmetry, it is called the \emph{quark mass matrix}. This matrix can be diagonalized by an $\SU(N_f)$ transformation, i.e. a flavor rotation, to become
    \begin{equation}
        m=\text{diag}(m_1,\dots,m_{N_f}).
    \end{equation}

    Classically and with $m=0$ the global symmetry should be $\SU(N_f)\times\U(1)_B\times\U(2)_R$. The mass terms and instanton corrections breaks $\U(1)_R$ of the $R$-symmetry, leaving the compact component $\SU(2)_R$ unbroken. The lagrangian should be invariant under the latter, it is a necessary and sufficient condition to have $\mN=2$ supersymmetry. Under the unbroken $\SU(2)_R$, the bosonic fields of the vector multiplet, i.e. $A_\mu,\phi,D,F$ are singlets butthe fermions form a doublet $(\lambda_\alpha,\psi_\alpha)$. Similarly, for the hypermultiplets, the fermions $\psi_{1\alpha},\bar{\psi}_{2\dalpha}$ are singlets while their scalar superpartners for a doublet $(H_1,\bar{H_2})$. The $\SU(2)_R$ symmetry cannot be made manifest in terms of $\mN=1$ sueprfields but the symmetry $\U(1)_J\subset\SU(2)_R$ is manifest in \eqref{eq:lag}.
    
    The selection rules resulting from the breaking of the classical symmetries by mass terms and instanton corrections can be describe bt assigning symmetry transformation properties to the corresponding parameters in the action. In particular, the quark mass matrix $m$ can be decomposed into a trace part $m_S$ that transforms as a singlet under $\SU(N_f)$ and a traceless part $m_A$ that transforms in the adjoint of $\SU(N_f)$. We summarize all the representations in which the fields and the parameters transform transform in table \ref{table:fieldrepr}.

    \begin{table}[H]
        \centering
        $
        \begin{array}{c|ccccc}
            & \SU(N_c) & \SU(N_f) & \U(1)_B & \U(1)_R & \U(1)_J \\ \hline
            \Phi & \textbf{adj} & \boldsymbol{1} & 0 & 2 & 0 \\
            Q & \boldsymbol{N_c} & \boldsymbol{N_f} & 1 & 0 & 1 \\
            \tilde{Q} & \bar{\boldsymbol{N_c}} & \bar{\boldsymbol{N_f}} & -1 & 0 & 1 \\
            m_A & \boldsymbol{1} & \textbf{adj} & 0 & 2 & 0 \\
            m_S & \boldsymbol{1} & \boldsymbol{1} & 0 & 2 & 0 \\
            \Lambda^{2N_c-N_f} & \boldsymbol{1} & \boldsymbol{1} & 0 & 2(2N_c-N_f) & 0
        \end{array}
        $
        \caption{Field representations.}
        \label{table:fieldrepr}
    \end{table}

    For $\mN=2$ theories, the $\beta$ function is exact at $1$-loop and $\beta_{1\text{-loop}}\propto 2N_c-N_f$. If if $N_f<2N_c$, the $\beta$-function is negative. The theory is asymptotically free and it generates a strong-coupling scale $\Lambda$. The instanton factor is proportional to $\Lambda^{2N_c-N_f}$ and the $\U(1)_R$ symmetry is anomalous. It is broken down to a discrete $\Z_{2N_f-N_c}$ symmetry. For $N_f=2N_c$, the theory is scale invariant and $\U(1)_R$ symmetry is not anomalous. No strong-coupling scale is generated and the theory is described in terms of its bare couplings.

    $D,F,F_1$ and $F_2$ are auxiliary fields and their equations of motion are:
    \begin{align}
        F^a_b &= \pdv{W}{\phi^b_a} = \sqrt{2}(H_2)^a_i (H_1)^i_b\\
        (F_1)^a_i &= \pdv{W}{(H_{1})^i_a} = \sqrt{2}(H_2)^b_i\phi^a_b + \sqrt{2}m^j_i(H_2)^a_j,\\
        (F_2)^i_a &= \pdv{W}{(H_{2})^a_i} = \sqrt{2}\phi^b_a (H_1)^i_b + \sqrt{2}m^i_j(H_1)^j_a,\\
        D^A &= -[\phi,\phi^\dagger]^A + \bar{H}_1T^AH_1-\bar{H}_2T^AH_2
    \end{align}
    where $T^A$ are the generators of $\SU(N_f)$ and $A=1,\dots,N^2_f-1$. Note that we can also integrate out the auxiliary fields $F_1$ and $F_2$ to recast the scalar potential for the hypermultiplets as a D-term contribution. The potential reads
    \begin{align}
        V(\phi,H_1,H_2) &= \frac{1}{2}\tr(D^AD_A)+\bar{F}F+\bar{F_1}F_1+\bar{F_2}F_2\\
        &= \frac{1}{2}\tr([\phi,\phi^\dagger]^2)+\frac{1}{2}\abs{\bar{H}_1T^AH_1-\bar{H}_2T^AH_2}^2\\
        &\qquad+2\abs{(H_2)^b_i\phi^a_b+m^j_i(H_2)^a_j}^2+2\abs{\phi^b_a (H_1)^i_b+m^i_j(H_1)^j_a}^2
    \end{align}

\section{Classical moduli space}

    The D-term equations are
    \begin{align}
        D:
        \begin{cases}
            \hspace{3.1cm}[\phi,\phi^\dagger]  &= 0 \\
            (H_1)^i_a(H^\dagger_1)^b_i-(H^\dagger_2)^i_a(H_2)^b_i &= \nu\delta^a_b
        \end{cases}
    \end{align}
    and the F-term equations are
    \begin{align}
        F:
        \begin{cases}
            \hspace{1.3cm}(H_1)^i_a(H_2)^b_i &= \rho\delta^b_a \\
            (H_1)^j_am^i_j+\phi^b_a(H_1)^i_b &= 0 \\
            m^i_j(H_2)^a_j+(H_2)^b_i\phi^a_b &= 0
        \end{cases}
    \end{align}\todo{Verify how to obtain these equation from the F-terms and D-terms}
    where $\nu$ and $\rho$ are arbitrary complex numbers.\todo{From where do those come from ?} The the two equations in the D-terms appear separately is a consequence of $\mN=2$ supersymmetry. One can square the D-term and show that the cross-term cancels or by noting that the first term is an $\SU(2)_R$-singlet and that that the second is part of a triplet\footnote{More generally, we will need to quotient by the complexified gauge transformation, which can be used to diagonalize $\phi$ and the first equation is automatically satisfied. This is another explanation.}.
    
    These equations suggest that $\phi,H_1$ and $H_2$ may get VEVs, which we denote by $\vev{\phi},\vev{H_1}$ and $\vev{H_2}$ respectively. Since there $N^2_c-1$ components $\phi^a_b$, $N_c\cdot N_f$ components $(H_1)^i_a$ and $N_c\cdot N_f$ components $(H_2)^a_i$, there are $N_c(N_c+2N_f)-1$ complex scalars in total. Meaning that the D-term and F-term equations define a subspace of $\C^{N_c(N_c+2N_f)-1}$. The \emph{classical moduli space} is defined as
    \begin{equation}
        \M_c\equiv Z(F,D)/G\subset \C^{N_c(N_c+2N_f)-1}
    \end{equation}
    where $G=\SU(N_c)$ is the gauge group. It turns out that we can just consider the F-term equations if we quotient by the complexified gauge group:
    \begin{equation}
        \M_c = Z(F)/G_\C.
    \end{equation}

    The solutions to those equations fall into various branches corresponding to the phases of the theory. The \emph{Coulomb branch} is the region of the moduli space where only the scalars from the vector multiplet can take VEVs, i.e. where $\vev{H_1}=\vev{H_2}=0$. The \emph{Higgs branch} is the region of the moduli space where only the scalars from the hypermultiplets can take VEVs, i.e. where $\vev{\phi}=0$. \emph{Mixed branches} are regions where all VEVs are non-vanishing. For simplicity we will mostly consider the case with no mass: $m^i_j=0$.

    \subsection{Coulomb branch}

        The only non-trivial equation is the first D-term equation $[\phi,\phi^\dagger]=0$, the other four are automatically satisfied. This equation is if and only $\phi$ belongs to $\mathfrak{h}_\C$, the complexified Cartan subalgebra of $\mathfrak{su}(N_c)$. In our case, this means that the scalar fields matrix $\phi$ can be diagonalized using a color rotation and put in the form
        \begin{equation}
            \phi = \sum_I \phi_Ih^I
        \end{equation}
        where $h^I=E_{I,I}-E_{I+1,I+1}$ with $(E_{I,J})_{ab}=\delta_{aI}\delta{bJ}\equiv$ are the generators of the Cartan subalgebra and $I=1,\dots N_c-1$ ($N_c-1$ is the rank of $\mathfrak{su}(N_c)$). In simpler words, the vacuum configurations are of the form
        \begin{equation}
            \phi=\text{diag}(\phi_1,\dots,\phi_{N_c}),\qquad \sum^{N_c}_{a=1}\phi_a=0.\label{eq:diagform}
        \end{equation}
        The vacuum configurations then depend on $N_c-1$ complex numbers so the Coulomb branch is a quotient of $\C^{N_c-1}$.
        
        At a generic point, the gauge group is broken to $\U(1)^r\times W$, where $W_G$ is the Weyl group of the gauge group, the group of residual gauge symmetries, while acting on $\phi$, do not not take it out of the Cartan subalgebra, i.e. keeps it the form \eqref{eq:diagform}. The low energy dynamic is the that of $r$ massless vector multiplets and $\dim G-r$ massive ones, with masses depending on the specific VEV's. The Weyl group of $\SU(N_c)$ is $S_{N_c-1}$. At last, the classical Coulomb branch is
        \begin{equation}
            \boxed{\M^V_c=\frac{\C^{N_c-1}}{S_{N_c-1}}.}
        \end{equation}
        A natural set of $\U(1)^{N-1}\times S_{N-1}$ invariant coordinates on this $(N_c-1)$-dimensional Coulomb branch can be shown to be
        \begin{equation}
            u_2=\sum_{i<j}\phi_i\phi_j,\quad u_3=\sum_{i<j<k}\phi_i\phi_j\phi_k,\quad \dots,\quad u_{N_c}=\phi_1\dots \phi_{N_c}, \qquad i,j,k=1,\dots,N_c.
        \end{equation}
        It has an orbifold singularity along submanifolds where some of the $\phi_a$'s are equal. In this case, some of the non-abelian gauge symmetry is restored. The scalar potential gives the mass of the fields $H_1$ and $H_2$ as $\phi_a+m_i$. The vanishing of these masses describes a complex co-dimension $1$ submanifold of the Coulomb branch. 

    \subsection{Higgs branch}

        Since we consider a vanishing quark mass matrix, only the second D-term equation and the first F-term equation are non-trivial. Recall that the squark fields $H_1$ and $H_2$ are complex matrices of size $N_c\times N_f$ and $N_f\times N_c$ respectively:
        \begin{equation}
            H_1=
            \begin{bmatrix}
                (H_1)^1_1 & \dots & (H_1)^{N_f} \\
                \vdots & & \vdots \\
                (H_1)^1_{N_c} & \dots & (H_1)^{N_f}_{N_c}
            \end{bmatrix},\qquad
            (H_2)^t=
            \begin{bmatrix}
                (H_2)^1_1 & \dots & (H_2)^{N_f} \\
                \vdots & & \vdots \\
                (H_2)^1_{N_c} & \dots & (H_2)^{N_f}_{N_c}
            \end{bmatrix}.
        \end{equation}

        \subsubsection{Squark VEV solutions}

            \begin{itemize}
                \item \underline{$N_f\geq2N_c$:} any solution can be put using flavor and color rotations:
                \begin{align}
                    \begin{split}
                    H_1 &= 
                    \begin{bmatrix}
                        \kappa_1 & & & 0 & & & 0 & \\
                        & \ddots & & & \ddots & & & \ddots \\
                        & & \kappa_{N_c} & & & 0 & & 
                    \end{bmatrix},\\
                    (H_2)^t &= 
                    \begin{bmatrix}
                        \tilde{\kappa}_1 & & & \lambda_1 & & & 0 & \\
                        & \ddots & & & \ddots & & & \ddots \\
                        & & \tilde{\kappa}_{N_c} & & & \lambda_{N_c} & & 
                    \end{bmatrix}
                \end{split}\label{eq:Higgsbranchsol}
                \end{align}
                where
                \begin{align}
                    \kappa_a\tilde{\kappa}_a &= \rho,\qquad\rho\in\C \label{eq:Higgsbrachcdt1}\\
                    \lambda^2_a &= \kappa^2_a-\frac{\abs{\rho}^2}{\kappa^2_a}+\nu,\qquad\nu\in\R \label{eq:Higgsbrachcdt2}
                \end{align}
                and the $\kappa_a's$ are non-zero if $\rho$ is non-zero.
                \item \underline{$N_f<2N_c$:} starting from a solution for $N_f=2N_c$ with some vanishing flavor columns, one can always construct a solution for $N_f<2N_c$ by removing those columns. On the other hand, starting from a solution for $N_f<2N_c$, one can always add vanishing flavor columns th construct a solution for $N_f=2N_c$. The necessary flavor rotation to put the solution into the form \eqref{eq:Higgsbranchsol} can be chosen not to act on these extra columns of zeros. This ensures us that this column-reduction procedure from $N_f=2N_c$ solutions will generate an $N_f<2N_c$ solution in every flavor orbit.
                
                To reduce \eqref{eq:Higgsbranchsol} by $2N_c-N_f$ columns, we must set $2N_c-N_f$ parameters to zero: $\lambda_1=\dots=\lambda_i=\kappa_1=\dots=\kappa_j=0$ with $i+j=2N_c-N_f$. By \eqref{eq:Higgsbrachcdt1}-\eqref{eq:Higgsbrachcdt2}, if some $\kappa$'s vanish, we must set $\rho=0$ before, which implies that some $\lambda_a$'s vanish too. Consequently, there are two possibilities to reducing columns, hence defining two sub-branches of the Higgs branch:
                \begin{itemize}[label=$\triangleright$]
                    \item \emph{baryonic branch}: only some $\lambda_a$'s vanish, more precisely, $i=2N_c-N_f$ and $j=0$. The VEV's have the form
                    \begin{align}
                        \begin{split}
                        H_1 &= 
                        \begin{bmatrix}
                            \kappa_1 & & & & & & \phantom{\lambda_1} & & \\
                            & \ddots & & & & & & \phantom{\ddots} & \\
                            & & \kappa_{N_f-N_c} & & & & & & \phantom{\lambda_{N_f-N_c}} \\
                            & & & \kappa_0 & & & & & \\
                            & & & & \ddots & & & & \\
                            & & & & & \kappa_0 & & &
                        \end{bmatrix},\qquad\kappa_a\in\R^+\\
                        (H_2)^t &= 
                        \begin{bmatrix}
                            \kappa_1 & & & & & & \lambda_1 & & \\
                            & \ddots & & & & & & \ddots & \\
                            & & \kappa_{N_f-N_c} & & & & & & \lambda_{N_f-N_c} \\
                            & & & \tilde{\kappa}_0 & & & & & \\
                            & & & & \ddots & & & & \\
                            & & & & & \tilde{\kappa}_0 & & &
                        \end{bmatrix},\qquad\lambda_a\in\R^+\\
                    \end{split}\label{eq:barynoicbranch}
                    \end{align}
                    where
                    \begin{align}
                        \kappa_a\tilde{\kappa}_a &= \rho,\qquad\rho\in\C\\
                        \lambda^2_a &= \kappa^2_a-\kappa^2_0+\abs{\rho}^2\left(\frac{1}{\kappa^2_a}-\frac{1}{\kappa^2_0}\right),\qquad\nu\in\R 
                    \end{align}
                    We use the term baryonic branch for the $N_f\geq 2N_c$ solutions \eqref{eq:Higgsbranchsol} as well. The baryonic branch exists for $N_f\geq N_c$. \todo{Is this case possible ?} One can see that the opposite case, i.e. taking only $\kappa_a$'s to vanish, with $i=0$ and $j=2N_c-N_f$, leads to a submanifold of the same branch uppon interchanging $H_1$ and $H_2$, which is a symmetry (charge conjugation) of our theory.
                    \item \emph{non-baryonic branch}: both some $\lambda_a$'s and some $\kappa_a$'s vanish, more precisely $i,j\neq0$ such that $i+j=2N_c-N_f$. From the constraints \eqref{eq:Higgsbrachcdt1}-\eqref{eq:Higgsbrachcdt2}, this implies that $\rho=\nu=0$ and $\kappa_a=\lambda_a$. The VEVs have the form
                    \begin{align}
                        \begin{split}
                            H_1 &= 
                            \begin{bmatrix}
                                \kappa_1 & & & 0 & & & 0 & \\
                                & \ddots & & & \ddots & & & \ddots \\
                                & & \kappa_r & & & 0 & & \\
                                & & & & & & & \\
                                & & & & & & &
                            \end{bmatrix},\\
                            (H_2)^t &= 
                            \begin{bmatrix}
                                0 & & & \kappa_1 & & & 0 & \\
                                & \ddots & & & \ddots & & & \ddots \\
                                & & 0 & & & \kappa_r & & \\
                                & & & & & & & \\
                                & & & & & & &
                            \end{bmatrix},\qquad\kappa_a\in\R^+\\
                        \end{split}\label{eq:nonbarynoicbranch}
                    \end{align}
                    where $r\leq \lfloor N_f/2\rfloor$ and $2N_c-N_f$ columns of zeros should be deleted by the column-reduction procedure. If $N_f$ is odd, there remains at least one column of zeros in the reduced matrices. The different values of $r$ give distinct submanifolds of the branch with maximal value. Nonetheless, we will refer to them as different baryonic branches. Some non-baryonic branches can also be obtained as submanifolds of the baryonic branch by setting $\rho=\kappa_0=\tilde{\kappa}_0=0$ in \eqref{eq:barynoicbranch}. The reason for these choices of terminology will become clear latter. Non-baryonic branches exist for $N_f\geq2$. For $N_f<2$ there is no Higgs branch at all.
                \end{itemize}
            \end{itemize}
        
        \subsubsection{Gauge symmetry and separate branches}

            Let us clarify the intersection pattern of the Higgs branches. We say that two Higgs branches are \emph{separate} if any path between the two goes through a point of enhanced gauge symmetry. This implies in particular that branches that if a branch has a larger unbroken gauge group than the other, they separate.

            \underline{Baryonic branch:} the $N_f\geq2N_c$ solution \eqref{eq:Higgsbranchsol} and the $N_f\leq2N_c$ solution \eqref{eq:barynoicbranch} completely break the gauge symmetry. By the Higgs mechanism, the number of massless supermultiplets is $\H=N_fN_c-N^2_c+1$. This counts the quaternionic dimension of the Higgs branch. There are submanifolds of the baryonic branch where the gauge symmetry is enhanced. These occur when two or more rows $H_1$ and $H_2$ vanish, i.e. if $\rho=\nu=0$ for \eqref{eq:Higgsbranchsol} and if $\rho=\kappa_0=0$ for \eqref{eq:barynoicbranch}, giving rise to non-baryonic branch VEV's \label{eq:nonbarynoicbranch} with
            \begin{equation}
                r\leq \min\{N_f-N_c,N_c-2\}.\label{eq:rrange}
            \end{equation}

            \underline{Non-baryonic branch:} there are non-baryonic branches with $r$ outside of the range \eqref{eq:rrange}. In general, the unbroken gauge group is $\SU(N_c-r)$ with $N_f-2r$ massless hypermultiplets in the fundamental. There are different unbroken gauge groups for different values of $r$ so they are separate branches. Higgs mechanism gives $\H=r(N_f-r)$ massless multiplets neutral under the unbroken gauge group.

        \subsubsection{Flavor symmetry}

            To identify the unbroken global symmetries on the Higgs branches, it is useful to define a basis of gauge-invariant quantities made from the squark VEV's:
            \begin{align}
                M^i_j &\equiv (H_2)^a_j(H_1)^i_a\\
                B^{i_1\dots i_{N_c}} &\equiv \eps^{a_1\dots a_{N_c}}(H_1)^{i_1}_{a_1}\dots(H_1)^{i_{N_c}}_{a_{N_c}}\\
                \tilde{B}_{i_1\dots i_{N_c}} &\equiv \eps_{a_1\dots a_{N_c}}(H_2)^{a_1}_{i_1}\dots(H_2)^{a_{N_c}}_{i_{N_c}}.
            \end{align}
            $M$ is called the \emph{meson field} and $B,\tilde{B}$ are called the \emph{baryon fields}. The latter are only defined for $N_f\geq N_c$.

            \underline{The baryonic branch:} on this branch, the baryonic fields are non-vanishing: $B,\tilde{B}\neq0$, hence the name, and from \eqref{eq:Higgsbranchsol} or \eqref{eq:barynoicbranch}, the meson field is
            \begin{equation}
                M=
                \begin{bmatrix}
                    \rho & & & \kappa_1\lambda_1 & & & 0 & \\
                    & \ddots & & & \ddots & & & \ddots \\
                    & & \rho & & & \kappa_{N_c}\lambda_{N_c} & & \\
                    & & & & & & & \\
                    & & & & & & &
                \end{bmatrix}\label{eq:baryonicbranchmesonfield}
            \end{equation}
            where the $\rho$-block is $N_c\times N_c$. For $N_f\leq 2N_c$ we should remove the appropriate number of columns from the right and rows from the bottom.
            
            For $N_f\geq 2N_c$, the meson field \eqref{eq:baryonicbranchmesonfield} and the non-vanishing baryon VEV's imply that the global symmetry is broken as
            \begin{equation}
                \SU(N_f)\times\U(1)_B\times\SU(2)_R\to\U(N_f-2N_c)\times\U(1)^{N_c-1}\times\SU(2)'_R.
            \end{equation}
            The number of real Goldstone boson bosons is then $\G=4N_fN_c-N^2_c-N_c+1$. Since the number of real parameters describing the Higgs branch in \eqref{eq:Higgsbranchsol} is $\mP=N_c+3$, we can see that $\G+\mP=4\H$. This is a check that we have a complete parametrization of this branch.

            For $N_C\leq N_f< 2N_c$, the global symmetry is broken as
            \begin{equation}
                \SU(N_f)\times\U(1)_B\times\SU(2)_R\to\SU(2N_c-N_f)\times\U(1)^{N_c-N_c}\times\SU(2)'_R.
            \end{equation}
            The number of real Goldstone boson is then $\G=-4N^2_c+4N_cN_f-N_f+N_c+1$. The number of real parameters describing the baryonic branch is $\mP=N_f-N_c+3$ and $\G+\mP=4\H$.

            \underline{The non-baryonic branches:} on these branches, the baryonic field vanishes, $B=\tilde{B}=0$, hence their name, and the meson field is given by
            \begin{equation}
                M=
                \begin{bmatrix}
                    0 & & & \kappa^2_1 & & & 0 & \\
                    & \ddots & & & \ddots & & & \ddots \\
                    & & 0 & & & \kappa^2_r & & \\
                    & & & & & & & \\
                    & & & & & & &
                \end{bmatrix}\label{eq:nonbaryonicbranchmesonfield}
            \end{equation}
            where the first block of zeros is $r\times r$. This implies that the global symmetry is broken as
            \begin{equation}
                \SU(N_f)\times\U(1)_B\times\SU(2)_R\to\U(N_f-2r)\times\U(1)^{r}\times\SU(2)'_R.
            \end{equation}
            The number of real Goldstone bosons is $\G=r(4N_f-4r-1)$ and $\mP=r$ so $\G+\mP=4\H$.

        \subsubsection{Gauge-invariant description}

            The configuration \eqref{eq:Higgsbranchsol} is sent to inequivalent points in the moduli space, but with the same physics, by global symmetry transformations. Gauge symmetry transformations on the other hand, sends them to equivalent point in the moduli space, which is not manifest in our writing. We want to describe the moduli space in terms of gauge-invariant coordinates, i.e. describe the various branches in terms of constraints on the meson field and the baryonic fields.

            The Higgs branch is a hyperKähler quotient of the squark space by the gauge group, with the D-terms and F-terms as moment maps. It is easier to work with a Kähler quotient, thus consider the theory as an $\mN=1$ theory with a superpotential interaction. In a Kähler quotient, the D-term equations are equivalent to quotienting by the complexified gauge group. This can be achieved by expressing the VEVs directly in terms of holomorphic gauge-invariant coordinates, such as the meson and baryonic fields, and by imposing the F-term equations. The non-trivial structure of the quotient is manifest in the fact that the gauge invariant coordinates are not independent as functions of the squark fields but they satisfy a set of polynomial relations which we must impose as constraints. Our goal is to find a set of generators of for these constraints and the F-term equations.

            By definition, the meson field $M$ and the baryonic fields $B,\tilde{B}$ must satisfy
            \begin{equation}
                B^{i_1\dots i_{N_c}}\tilde{B}_{j_1\dots j_{N_c}}=M^{[i_1}_{j_1}\dots M^{i_{N_c}]}_{j_{N_c}}
            \end{equation}
            which can be rewritten as
            \begin{equation}
                (\star B)\tilde{B}=\star(M^{N_c})\label{eq:constraint1}
            \end{equation}
            with $(\star B)_{i_{N_c+1}\dots i_{N_f}}=\eps_{i_1\dots i_{N_f}}B^{i_1\dots i_{N_c}}$.
            
            Also, since any expression antisymmetrized on $N_c+1$ color indices must vanish, any product of $M$'s, $B$'s and $\tilde{B}$'s antisymmetrized on $N_c+1$ upper or lower indices must vanish. For $B,\tilde{B}\neq0$, an induction argument shows that the constraint \eqref{eq:constraint1} together with
            \begin{equation}
                M\cdot\star B=M\cdot\star\tilde{B}=0\label{eq:constraint2}
            \end{equation}
            where $\cdot$ represents the contraction of flavor indices. If $B=\tilde{B}=0$, all the other constraints are automatically satisfied and \eqref{eq:constraint1} implies \eqref{eq:constraint2}

            From \eqref{eq:constraint1} and \eqref{eq:constraint2}, one can show that
            \begin{equation}
                \text{rank}(M)\leq N_c.
            \end{equation}

            The first F-terms gives two new constraints:
            \begin{align}
                M'\cdot B = \tilde{B}\cdot M' &= 0\label{eq:constraint3}\\
                M\cdot M' &= 0\label{eq:constraint4}
            \end{align} 
            and the other two equations are relevant only for mixed branches. Finally, a complete set of constraints is given by \eqref{eq:constraint1},\eqref{eq:constraint2},\eqref{eq:constraint3} and \eqref{eq:constraint4}.

            The condition \eqref{eq:constraint4} is already quite restrictive; its only solutions are, up to flavor rotations, the meson field configuration \eqref{eq:baryonicbranchmesonfield} and \eqref{eq:nonbaryonicbranchmesonfield}. NThe non-baryonic solutions have rank $r\leq\lfloor N_f/2\rfloor$. For $N_f> 2N_c$, this will be reduced to $r\leq N_c$ by \eqref{eq:constraint2}. For $N_f\leq 2N_c$ on the other hand this constraint is automatically satisfied and \eqref{eq:constraint2} is implied by \eqref{eq:constraint4}.

    \subsection{Mixed branches}

        

\section{Quantum moduli space}

\section{Quantum Higgs branches and the non-renormalization theorem}

\section{Higgs branch roots}

    \subsection{Non-baryonic root}

    \subsection{Baryonic root}

\part{$A_1$ singularity}

\part{$A_n$ singularity}

\part{$\D_4$ singularity}

\appendix

\section{References guide}

\begin{itemize}
    \item General strings and D-branes: \cite{DbranespartI},\cite{DbranespartII},\cite{notesDbranes}
    \item He: review: \cite{he2004lectures}, thesis: \cite{masterHe}
    \item Orbifold construction for $\Gamma\subset\SU(2)$: type $A$ : \cite{douglas1996dbranes}, type $D$ and $E$: \cite{PhysRevD.55.6382}
    \item Orbifold construction for $\Gamma\subset\SU(3)$:\cite{Hanany_1999}
    \item Orbifold construction for $\Gamma\subset\SU(4)$:\cite{Hanany:1999sp}
    \item Formalization of projection to daughter theories: \cite{vafa1998},\cite{silervstein1998}
    \item Quivers representations and varieties: \cite{brion},\cite{kirillov2016quiver}
    \item On toric varieties: \cite{cox2011toric},\cite{torigeomandCY}(\cite{fulton1993introduction},\cite{oda1988convex})
    \item Forward algorithm  for toric singularities devlopments: \cite{FA1},\cite{FA2},\cite{FA3},\cite{FA4},\cite{FA5}, formalization: \cite{FA6},\cite{FA7},\cite{FA8}
    \item Toric diagrams, dimer diagrams and Higgsing: \cite{Argurio_20081}
    \item Fractional branes: \cite{Argurio_20082}
    \item Formalization of inverse algorithm for toric singularities: \cite{Feng_2001}
    \item geometry and K3 surfaces: \cite{https://doi.org/10.48550/arxiv.hep-th/9611137}
    \item general review of SYM and their brane description: \cite{Elitzur_1997}
    \item Hanany-Witten setup: \cite{1997}
    \item geometric engineering: \cite{https://doi.org/10.48550/arxiv.hep-th/9706110},\cite{Katz:1996xe},\cite{Katz:1996fh}
    \item link between graph theory and Yang-Mills (constructed from string theory with the three different methods) to study the finiteness of the theories: \cite{Hanany:1999sp}
\end{itemize}

%\section{Properties of D-branes}

    \subsection{SYM from D-branes}

        The dynamics of D-branes is described by the Dirac-Born-Infeld action
        \begin{equation}
            S_{\text{DBI}}[X,F] = -\frac{T_p}{g_s}\int\d^{p+1}\sigma~\sqrt{-\det\limits_{0\leq a,b\leq p}(\eta_{ab}+\p_a X^m\p_b X_m+2\pi\alpha'F_{ab})}.
        \end{equation}
        The latter can be expended for slowly-varying fields, which is equivalent to passing to the field theory limit $\alpha'\to0$. The resulting action is the action of a $\U(1)$ gauge theory in $p+1$ dimensions with $9-p$ real scalar fields. This action is exactly the same than the one we would obtain by dimensionally-reducing a pure $\U(1)$ Yang-Mills gauge theory in 10 spacetime dimensions with the identification
        \begin{equation}
            g_{\text{YM}}=g_sT^{-1}_p(2\pi\alpha')^{-2}=\frac{g_s}{\sqrt{\alpha'}}(2\pi\sqrt{\alpha'})^{p-1}.
        \end{equation}

        This construction can be generalized for multiple D-branes. It now results in a non-abelian theory. The general statement is the following:
        \begin{result}
            The low-energy dynamics of $N$ parallel, coicident D$p$-branes in flat space is described in static gauge by the dimensional reduction to $p+1$ dimensions of pure $10d$ $\mN=1$ supersymmetric Yang-Mills theory with gauge group $\U(N)$ in ten spacetime dimensions.
        \end{result}
        Recall that the $10$-dimensional action is given by
        \begin{equation}
            S_{\text{YM}} = \frac{1}{4g^2_{\text{YM}}}\int\d^{10}x~\left[ \tr(F_{\mu\nu}F^{\mu\nu})+2i\tr(\bar{\psi}\Gamma^\mu D_\mu\psi)\right],\label{eq:SYMaction}
        \end{equation}
        where $F_{\mu\nu}=\p_\mu A_\nu-\p_\nu A_\mu+i[A_\mu,A_\nu]$ is the non-abelian field strength of the $\U(N)$ gauge field $A_\mu$, $D_\mu=\p_\mu-i[A_\mu,\psi]$, $\Gamma^\mu$ are $16\times 16$ Dirac matrices \todo{Why ?}, and the $N\times N$ Hermitian fermion field $\psi$ is a $16$-component Majorana-Weyl spinor of the Lorentz group $\SO(1,9)$ which transforms under the adjoint representation of the gauge group $\U(N)$. On-shell, there are eight on-shell bosonic, gauge field degrees of freedom, and eight fermionic degrees of freedom, after imposition of the Dirac equation $\xout{D}\psi=\Gamma^\mu D_\mu\psi=0$. One can verify that this action is invariant under the supersymmetry transformations
        \begin{align*}
            \delta_\eps A_\mu &= \frac{i}{2}\bar{\eps}\Gamma_\mu\psi,\\
            \delta_{\eps}\psi &= \frac{1}{2}F_{\mu\nu}[\Gamma^\mu,\Gamma^\nu]\eps,
        \end{align*}
        where $\eps$ is an Majorana-Weyl spinor.

        Using \eqref{eq:SYMaction}, we can construct a supersymmetric Yanf-Mills gauge theory in $p+1$ dimensions with $16$ independent supercharges by dimensional reduction: we take all fields to be independent of the coordinates $X^{p+1},\dots, X^9$, then the ten-dimensional gauge field $A_\mu$ splits into a $(p+1)$-dimensional $\U(N)$ gauge field $A_a$ plus $9-p$ Hermitian scalar fields $\Phi^m=X^m/2\pi\alpha'$ in the adjoint representation of $\U(N)$. The D$p$-brane action is thereby obtained from the dimensionality reduced field theory as
        \begin{equation}
            S_{\text{D}p} = -\frac{T_pg_s(2\pi\alpha')^2}{4}\int\d^{p+1}\sigma~\tr\left(F_{ab}F^{ab}+2D_a\Phi^m D^a\Phi_m+\sum_{m\neq n}[\Phi^m,\Phi^n]^2+\text{fermions}\right)\label{eq:SDp}
        \end{equation}
        where $a,b=0,\dots,p$, $m,n=p+1,\dots,9$. We do not explicitly display the fermionic contributions for the moment. In conclusion, the low-energy brane dynamics is described by a supersymmetric Yang-Mills theory on the D$p$-brane worldvolume which is dynamically coupled to the transverse, adjoint scalar fields $\Phi^m$.

        The scalar potential is given by
        \begin{equation}
            V(\Phi)=\sum_{m\neq n}[\Phi^m,\Phi^n]^2.
        \end{equation}
        It is negative definite because $[\Phi^m,\Phi^n]^\dagger=[\Phi^n,\Phi^m]=-[\Phi^m,\Phi^n]$. A classical vacuum of the field theory defined by \eqref{eq:SDp} corresponds to a static solution of the equations of motion whereby the potential energy of the system is minimized. It is given by the field configurations which solve simultaneously the quations $F_{ab}=D_a\Phi^m=\psi^a=0$ and $V(\Phi)=0$. Since all term in $V(\Phi)$ have the same sign, the equation $V(\Phi)=0$ is equivalent to the equation $[\Phi^m,\Phi^n]=0$ for all $m,n$ and at each point in the $(p+1)$-dimensional worldvolume of the branes. This implies that the $N\times N$ hermitian matrix fields $\Phi^m$ are simultaneously diagonalizable by a gauge transformation, so that we may write
        \begin{equation}
            \Phi^m=U
            \begin{bmatrix}
                X^m_1 & & & 0 \\
                & X^m_2 & & \\
                & & \ddots & \\
                0 & & & X^m_N
            \end{bmatrix}U^{-1},\label{eq:diagPhi}
        \end{equation}
        the matrix $U$ is independent of $m$. The simultaneous,
        real eigenvalues $X^m_i$ give the positions of the $N$ distinct D-branes in the $m$-th transverse direction. It follows that the moduli space of classical vacua for the $(p+1)$-dimensional field theory \eqref{eq:SDp} is the quotient space $(\R^{9-p})^N/S_N$, where the factors of $\R$ correspond to the positions of the $N$ D$p$-branes in the $(9-p)$-dimensional transverse space, and $S_N$ is the symmetric group acting by permutations of the $N$ coordinates $X_i$. The group $S_N$ corresponds to the residual Weyl symmetry of the $\U(N)$ gauge group acting in \eqref{eq:diagPhi}. It represents the permutation symmetry of a system of $N$ \emph{indistinguishable} D-branes.

        From \eqref{eq:SDp} one can easily deduce that the masses of the fields corresponding to the off-diagonal matrix elements are given precisely by the distances $\abs{x_i-x_j}$ between the corresponding branes. This description means that an interpretation of the D-brane configuration in terms of classical geometry is only possible in the classical ground state of the system, whereby the matrices $\Phi^m$ are simultaneously diagonalizable and the positions of the individual D-branes may be described through their spectrum of eigenvalues. This gives a simple and natural dynamical mechanism for the appearence of ``non-commutative geometry'' at short distances, where the D-branes cease to have well-defined positions according to classical geometry.

        \todo{The end of this section has to be rewritten}

    \subsection{D-branes and residual SUSY in type II theories}

        The minimal irreducible representation in 10 dimensions is a Majorana-Weyl representation of dimension 8. In type II theories, we have $\mN=(1,1)$ for IIA and $\mN=(2,0)$ for IIB. Because of the string origin of the generators, the two supersymmetry generators $\eps_L$ and $\eps_R$ (Majorana-Weyl spinors) satisfy
        \begin{equation}
            \eps_L=\Gamma_{11}\eps_L,\qquad \eps_R=\eta\Gamma_{11}\eps_R
        \end{equation}
        with $\eta=+1$ for IIB and $\eta=-1$ for IIA theory. For a D$p$-brane, the supersymmetry projections is the following:
        \begin{equation}
            \eps_L=\Gamma_0\dots\Gamma_p\eps_R.
        \end{equation}
        In other words, the supersymmetries with generators of the form
        \begin{equation}
            Q_\alpha+\Gamma_0\dots\Gamma_p\bar{Q}_{\dalpha}\label{eq:susypresved}
        \end{equation}
        are preserved by the D$p$-brane while the one with generators of the form
        \begin{equation}
            Q_\alpha-\Gamma_0\dots\Gamma_p\bar{Q}_{\dalpha}\label{eq:susybroken}
        \end{equation}
        are broken. They violate the boundary conditions. Since there is the same number of generators of the form \eqref{eq:susypresved} than of the form \eqref{eq:susybroken}, exactly haf of the supersymmetry is broken. The idea that one spacetime direction would break one supercharge could be reasonable if supersymmetries were transforming as vectors which not the case; supercharges transform as spinors. It would also be incompatible with the T-duality because two branes of different dimensions must have the same number of unbroken supercharges if there is a T-duality relating them: the number of unbroken supercharges is the same for all dual descriptions (a necessary condition for the equivalence). And indeed, in the correct theory, that's the case. Every type II D-brane breaks half of the supercharges.

        To obtain the previous relations, we start by the ones from M-theory and compactify the 11th direction, getting type IIA theory. $\Gamma_{11}$ then plays the role of the chiral projector in 10 dimensions; the supersymmetry parameters are related by $\eps_L=\frac{1}{2}(1+\Gamma_{11})\eps$ and $\eps_R=\frac{1}{2}(1-\Gamma_{11})\eps$. The relations for type IIB theory are then obtained by T-duality. Under a T-duality over the $\hat{i}$ direction, the supersymmetry parameters transform as
        \begin{align*}
            \eps_L &\mapsto \eps_L,\\
            \eps_R &\mapsto \Gamma_i\eps_R.
        \end{align*}
        The tension of a D$p$-brane is given by
        \begin{equation}
            T_{p} = \frac{1}{(2\pi)^pg_sl^{p+1}_s}.
        \end{equation}
        This completely fixes the Newton constant: the tension of electric-magnetic duals must satisfy:
        \begin{equation}
            T_pT_{D-p-4} = \frac{2\pi}{16\pi G_D}.
        \end{equation}
        In ten dimensions, this gives $G_{10}=8\pi^6g^2_sl^8_s$.

        The dualities are defined as follows:
        \begin{align*}
            \text{S-duality} &: g_s\mapsto\frac{1}{g_s},\qquad l^2_s\mapsto g_sl^2_s,\\
            \text{T-duality} &: R\mapsto\frac{l^2_s}{R},\qquad g\mapsto g_s\frac{l_s}{R}.
        \end{align*}

    \subsection{D-branes wrapping cycles}

        A D$p$-brane worldvolume $\phi:\Sigma\to X$ in spacetime $X$ \emph{wraps} a cycle $c\in H_{p+1}$ if the pushforward $\phi_*(\Sigma)\in H_\text{\textbullet}(X)$ of the fundamental class of $\Sigma$ is the class $[c]$ of the given cycle in $X$. If the pushforward is a mutliple of $[c]$, then the branes wraps $c$ multiple times.

\section{Algebraic geometry}

    \subsection{Elements}

        An important idea in algebraic geometry is that is is really the alegbra of function on it that defines a space. For affine varities $X$, this is illustrated by the fact that the structure of $X$ is really contained in its coordinate ring $K[x_1,\dots,x_n]/I(X)$ and by the isomorphism
        \begin{equation}
            K[x_1,\dots,x_n]|_X=K[x_1,\dots,x_n]/I(X).
        \end{equation}
        Now an algebraic set $Z(T)$ is irreducible if $I(Z(T))$ is prime. So there is a one-to-one correspondence between prime ideals and affine varieties.

        Given an algebraic variety, one can modify the equations continuously by varying some parameters and the variety will be ``deformed'' accordingly. It is called the \emph{variation of the omplex structure}. The space of all complex deformations of an affine variety $X$ is called the \emph{complex moduli space} of $X$. For a Calabi-Yau manifold, the linearization of the complex moduli space (tangent space) is given by the cohomology group $H^{m-1,1}(X)$, where $m$ is the dimension of $X$. In general, it is much more complicated.

    \subsection{Divisors and line bundles}

        A \emph{(Weyl) divisor} $D$ of a complex variety $X$ is a linear combination (formal sum with integer coefficients) of co-dimension one, irreducible subvarieties,
        \begin{equation}
            D=\sum_i n_iV_i,\qquad n_i\in\Z,V_i\subset X.
        \end{equation}
        It said to be effective if all $n_i\geq1$. To any line bundle $L$ with a regular section $s$ (that is, on any open subset $U_\alpha$, $s_\alpha=s|_{U_\alpha}$ is a polyomial in the local coordinates) we can associate a hypersurface $Y\subset X$ defined as
        \begin{equation}
            Y=\{p\in X|s(p)=0\}.
        \end{equation}
        This hypersurface $Y$ can then be decomposed into irreducible parts (affine patches) on which $s_\alpha$ can be factorized in $\C[x_1,\dots,x_n]$ and decomposed in prime ideals $P_i$ of multiplicity $n_i$. Assembling all the $V^\alpha_i$ otgether, we construct co-dimension one subvarities $V_i$ that can be used to form divisors. One can also proceed the other ay around and, given a divisor $D=\sum_i n_iV_i$, define a line bundle $\mathcal{O}_X(D)$ whose sections vanish on each $V_i$ with a zero of order $n_i$. This construction can be generalized to divisors with negative coefficients $n_i<0$ in which case now have poles of order $n_i$ in $V_i$.

    \subsection{Singularities and resolutions}

        A \emph{rational map} from a variety $X$ to another $Y$ is a morphism from a non-empty subset $U\subset X$ to $Y$. Recall that, by definition of the Zariski topology, a non-empty open subset is always dense. Concretely, a rational map can be written in coordinates using ration functions (quotient of polynomials). A \emph{birational map} is an invertible rational map. It induces an isomorphism between two non-empty open subsets. In this case, $X$ and $Y$ are said to be \emph{birationally equivalent}.

        The \emph{resolution of a singularity} of an algebraic variety $V$ is a non-singular variety $W$ with a proper birational map $W\to V$. For varities over fields of characteristic $0$, it was proven (Hironaka, 1964) that \todo{fill in}.

    \subsection{Projective plane curves}

        In $\C\P^2$, we consider a hypersurface defined by a single polynomial $p$ of degree $d$. If
        \begin{equation}
            \pdv{p(x)}{x_i}=0
        \end{equation}
        for all $i$ whenever $p(x)=0$, then the curve is said to be regular, it is a Riemann surface. The latter are classified by their genus and
        \begin{equation}
            g=\frac{(d-1)(d-2)}{2}.
        \end{equation}
        For $d=3$, the most geenral polynomial is
        \begin{equation}
            \sum_{i+j+k=3}c_{ijk}x^i_0x^j_1x^k_2=0
        \end{equation}
        and defines a torus, also called \emph{elliptic curve}. There $10$ independant parameters but $9$ of them can be removed by a $\GL(3,\C)$ transformation, leaving us with only one complex parameter; the complex structrue modulus of the torus.




    \todo{This section has to be rewritten.}

\section{Quivers in string theory and Yang-mills in graph theory}

%



\part{Toric singularities}

    The next best thing to orbifold singularities is the toric singularities.  The a specific class of supersymmetric gauge theories whose space of vacua is toric is called \emph{toric quiver gauge theories}. In this case, the inverse algorithm has been formalized in \cite{Feng_2001}.

\section{Gauged linear sigma model (GLSM)}

    Witten's gauged linear sigma model provides a physical perspective on toric varieties which provides us with the right approach for the forward algorithm. Let us consider the vectorspace $\C^q$ with complex coordinates $z_1,\dots,z_q$.
        

    \subsection{Calabi-Yau and non-compactness conditions}

\section{Correspondence between gauge theory and singularity}

    Above, we presented all the possible orbifold constructions of supersymmetric quiver gauge theories in four dimensions. We started from quotienting the transverse space and we found the corresponding (supersymmetric) gauge theory. In other words, we started from the singularity a found the gauge theory. We can therefore consider that the orbifold singularities are understood. However, not all singularities are orbifold one, such as the conifold for example. We can then ask ourselves how to obtain the gauge theory for more general singularities than the orbifold ones. Is a general approach possible ? On the other hand, we can also study the converse question; is it possible to to obtain the singularity from the gauge theory? And if it is, how so? In general we will see that there is a bijection between the four-dimensional supersymmetric worldvolume gauge theory and the Calabi-Yau singularity. We now detail this bijection.

    \begin{figure}[H]
        \centering
        \includegraphics[scale=0.3]{Pictures/algorithm.png}
        \caption{Inverse and forward algorithm, from \cite{he2004lectures}.}
    \end{figure}

    \subsection{From gauge theory to singularity: forward algorithm}

        We start with the simplest question: how to recover the singularity from the gauge theory? We already mentioned that the vacuum parameter space of the scalar fields of the gauge theory is the so-called moduli space, denoted $\M$. Because our D$3$-brane is a point in the Calabi-Yau threefold, the vacuum moduli space $\M$ is the affine coordinates of the Calabi-Yau singularity $S$.

        For the ADE $\mN=2$ theories discussed in section \ref{sec:N2QGT}, by the Kronheimer-Nakajima construction \cite{Kronheimer1990}, the moduli space is a hyper-Kähler quotient. In general, the moduli space can be constructed as a \emph{quiver variety}, i.e. a variety constructed from the moduli space of quiver a quiver representation. More rpecisely, given the dimensions of the vector spaces assigned to every vertex, one can form a variety which characterizes all representations of that quiver with those specified dimensions, and consider stability conditions. Let us see some examples of this.

        The anomaly free condition is
        \begin{equation}
            (a_{ij}-a_{ji})n_i=0.
        \end{equation}
        \todo{Explain more.}

    \subsection{Forward algorithm for abelina orbifolds}

        

    \subsection{From singularity to gauge theory: inverse algorithm}

        Mathematically, a quiver gauge theory is a representation of a finite quiver with relations. The labels are $\{N_i\in\Z_+\}$, they correspond to the dimension of the vector space $\{V_i\}$. The gauge group is $\prod_i\SU(N_i)$. The gauge fields are self-adjoint arrows $\Hom(V_i,V_i)$ while the matter fields are bi-fundamentals fermions/bosons and are arrows $X_{ij}\in\Hom(V_j,V_i)$. For a quiver with adjacency matrix $a_{ij}$, the gauge anomaly cancellation condition can be generally expressed as
        \begin{equation}
            (a_{ij}-a_{ji})N_i=0.
        \end{equation}
        At last, there are some relations that arises the superpotential $W(\{X_{ij}\})$. The vacuum is the minima of the superpotential. In other words,
        \begin{equation}
            \pdv{W}{X_{ij}}=0.
        \end{equation}

    \subsection{Application to Toric del Pezzo's}

        \begin{figure}
            \centering
            \includegraphics[scale=0.4]{Pictures/delPezzo.png}
        \end{figure}

\section{Toric duality}



%\include{Body/NonToric.tex}

%\include{Body/BeyongBraneProbes.tex}

%\section{Some finite subgroups}

    \subsection{Finite subgroups of $\SU(2)$ and $\SL(2,\C)$}\label{app:su2subgroups}

        \subsubsection{Finite subgroups}

            The first thing to recall is that every finite subgroup of $\SL(2,\C)$ is isomorphic to a subgroup of $\SU(2,\C)$ and vice-versa, so we equivalently talk about the subgroups of $\SU(2)$. The finite subgroups of $\SU(2)$, called the \emph{binary polyhedral groups}, are the doubles covers of the finite subgroups of $\SO(3)$ that are called \emph{polyhedral groups}. They simply constitutes the symmetries of the Platonic solids. The groups fall into two infinite series, associated to the regular polygons, as well as three exceptional, associated with the 5 regular polyhedra: the tetrahedron (self-dual), the cube (and its dual octahedron), the icosahedron (and its dual dodecahedron).

            More precisely, the finite subgroups of $\SL(2,\C)$ are
            \begin{itemize}
                \item $\Z_n$ : cyclic group of order $n$ ($n\geq2$) generated by
                \begin{equation}
                    \begin{bmatrix}
                        \zeta_m & 0\\
                        0 & \zeta^{-1}_m
                    \end{bmatrix}
                \end{equation}
                \item $2\D_n$ : \emph{binary dihedral groups} (also known as the \emph{dicyclic group}) of order $4n$ ($n\geq1$) generated by
                \begin{equation}
                    A \equiv
                    \begin{bmatrix}
                        \zeta_{2n} & 0\\
                        0 & \zeta^{-1}_{2n}
                    \end{bmatrix}\quad \text{ and }
                    B \equiv 
                    \begin{bmatrix}
                        0 & i\\
                        i & 0
                    \end{bmatrix}
                \end{equation}
                One can show that $A^n=B^2$ and that $AB=BA^{-1}$ so that $2\D_n=\{B^bA^a|0\leq b \leq 3, 0\leq a \leq n-1\}$. This rewriting of the most general element of the group will be useful.
                \item $2\mathcal{T}$ : \emph{binary tetrahedral group} of order $24$ generated by $D_2$ and
                \begin{equation}
                    C \equiv \frac{1}{\sqrt{2}}
                    \begin{bmatrix}
                        \zeta_8 & \zeta^3_8\\
                        \zeta_8 & \zeta^7_8
                    \end{bmatrix}
                \end{equation}
                \item $2\mathcal{O}$ : \emph{binary octahedral group} of order $48$ generated by $\mathcal{T}$ and
                \begin{equation}
                    D \equiv 
                    \begin{bmatrix}
                        \zeta^3_8 & 0\\
                        0 & \zeta^5_8
                    \end{bmatrix}
                \end{equation}
                \item $2\mathcal{I}$ : \emph{binary icosahedral group} of order $120$ generated by
                \begin{equation}
                    E \equiv -\frac{1}{\sqrt{5}}
                    \begin{bmatrix}
                        \zeta^4_5-\zeta_5 & \zeta^2_5-\zeta^3_5\\
                        \zeta^2_5-\zeta^3_5 & \zeta_5-\zeta^4_5
                    \end{bmatrix}\quad \text{ and }
                    F \equiv -\frac{1}{\sqrt{5}}
                    \begin{bmatrix}
                        \zeta^2_5-\zeta^4_4 & \zeta^4_5-1\\
                        1-\zeta_5 & \zeta^3_5-\zeta_5
                    \end{bmatrix}
                \end{equation}
            \end{itemize}
            with $\zeta_m\equiv e^{i\frac{2\pi}{m}}$ such that $(\zeta_m)^m=1$. Note that the orders are all divisible by $2$. This is because the center of $\SU(2)$ is $\Z_2$.

        \subsubsection{Irreducible representations}\label{sec:irrep}

            \begin{itemize}
                \item $\Z_n$ has $n$ irreducible representations. They are all $1$-dimensional (since $\Z_n$ is abelian) and are given by
                \begin{equation}
                    \rho_k(g)=\zeta^k_n
                \end{equation}
                with $k=0,\dots,n-1$.
                \item $2\D_n$ has $n+3$ irreducible representations: $4$ of dimension $1$ and $n-1$ of dimension $2$. The $1$-dimensional ones are given by
                \begin{equation*}
                \begin{array}{|c|c|c|c|}
                    \hline
                    n & \rho(A) & \rho(B) & \rho(B^bA^a) \\
                    \hline
                    \multirow[c]{4}{*}{\text{even}} & \multirow[c]{2}{*}{1} & 1 & 1 \\ \cline{3-4}
                    & & -1 & (-1)^b \\ \cline{2-4}
                    & \multirow[c]{2}{*}{-1} & 1 & (-1)^a \\ \cline{3-4}
                    & & -1 & (-1)^{a+b} \\
                    \hline
                    \multirow[c]{4}{*}{\text{odd}} & \multirow[c]{2}{*}{1} & 1 & 1 \\ \cline{3-4}
                    & & -1 & (-1)^b \\ \cline{2-4}
                    & \multirow[c]{2}{*}{-1} & i & (-1)^ai^b \\ \cline{3-4}
                    & & -i & (-1)^a(-i)^b \\
                    \hline
                \end{array}
                \end{equation*}
                and the $2$-dimensional ones are given binary by
                \begin{align*}
                    \rho_r(A) &= 
                    \begin{bmatrix}
                        e^{i\frac{\pi}{n}r} & 0\\
                        0 & e^{-i\frac{\pi}{n}r} 
                    \end{bmatrix}\\
                    \rho_r(B) &= 
                    \begin{bmatrix}
                        0 & -1 \\
                        1 & 0
                    \end{bmatrix}
                \end{align*}
                with $r=1,\dots,n-1$.
            \end{itemize}

        \subsubsection{Character tables}

            \begin{table}[H]
                \centering
                {\small
                \begin{equation*}
                        \begin{array}{|c|c|c|c|c|c|}
                            \hline
                            \text{conj. class repr.} & e & M & M^2 & \dots & M^{n-1} \\ \hline
                            \text{conj. class order} & 1 & 1 & 1 & \dots & 1 \\
                            \hline
                            V_0 & 1 & 1 & 1 & \dots & 1 \\
                            V_1 & 1 & \zeta_n & \zeta^2_n & \dots & \zeta^{n-1}_n \\
                            V_2 & 1 & \zeta^2_n & \zeta^4_n & \dots & \zeta^{2(n-1)}_n \\
                            V_3 & 1 & \zeta^3_n & \zeta^6_n & \dots & \zeta^{3(n-1)}_n \\
                            \vdots & \vdots & \vdots & \vdots & \ddots &  \\
                            V_{n-1} & 1 & \zeta^{(n-1)}_n & \zeta^{2(n-1)}_n & \dots & \zeta^{(n-1)^2}_n \\ \hline
                            W & 2 & 2\cos\left(\frac{2\pi}{n}\right) & 2\cos\left(\frac{4\pi}{n}\right) & \dots & 2\cos\left(\frac{2\pi(n-1)}{n}\right) \\ \hline 
                            \end{array}
                    \end{equation*}}
                \caption{Character table of $\Z_n$.}
            \end{table}

            \begin{table}[H]
                \centering
                {\small
                \begin{equation*}
                        \begin{array}{|c|c|c|c|c|c|c|c|c|}
                            \hline
                            \text{conj. class repr.} & e & B^2 & B & BA & A & A^2 & \dots & A^{n-1} \\ \hline
                            \text{conj. class order} & 1 & 1 & n & n & 2 & 2 & \dots & 2 \\
                            \hline
                            V_0 & 1 & 1 & 1 & 1 & 1 & 1 & \dots & 1 \\ 
                            V_1 & 1 & 1 & -1 & -1 & 1 & 1 & \dots & 1 \\ 
                            V_2 & 1 & 1 \text{ ou } -1 & 1 \text{ ou } i & -1 \text{ ou } -i & -1 & 1 & \dots & (-1)^{n-1} \\ 
                            V_3 & 1 & 1 \text{ ou } -1 & -1 \text{ ou } -i & 1 \text{ ou } i & -1 & 1 & \dots & (-1)^{n-1} \\
                            V_4 & 2 & -2 & 0 & 0 & 2\cos\frac{\pi}{n} & 2\cos\frac{2\pi}{n} & \dots & 2\cos\frac{(n-1)\pi}{n}\\
                            V_5 & 2 & 2 & 0 & 0 & 2\cos\frac{2\pi}{n} & 2\cos\frac{4\pi}{n} & \dots & 2\cos\frac{2(n-1)\pi}{n}\\
                            \vdots & \vdots & \vdots & \vdots & \vdots & \vdots & \vdots & \ddots & \vdots \\
                            V_{n+2} & 2 & 2(-1)^{n-1} & 0 & 0 & 2\cos\frac{(n-1)\pi}{n} & 2\cos\frac{2(n-1)\pi}{n} & \dots & 2\cos\frac{(n-1)^2\pi}{n} \\ \hline
                            W & 2 & -2 & 0 & 0 & 2\cos\left(\frac{\pi}{n}\right) & 2\cos\left(2\frac{\pi}{n}\right) & \dots & 2\cos\left(\frac{\pi}{n}(n-1)\right) \\ \hline
                            \end{array}
                    \end{equation*}}
                \caption{Character table of $2\D_n$.}
            \end{table}

            \begin{table}[H]
                \centering
                {\small
                \begin{equation*}
                        \begin{array}{|c|c|c|c|c|c|c|c|}
                            \hline
                            \text{conj. class repr.} & e & B^2 & B & C & C^2 & C^4 & C^5 \\ \hline
                            \text{conj. class order} & 1 & 1 & 6 & 4 & 4 & 4 & 4\\
                            \hline
                            V_0 & 1 & 1 & 1 & 1 & 1 & 1 & 1 \\
                            V_1 & 2 & -2 & 0 & 1 & -1 & -1 & 1 \\
                            V_2 & 3 & 3 & -1 & 0 & 0 & 0 & 0 \\
                            V_3 & 2 & -2 & 0 & e^{i\frac{2\pi}{3}} & -e^{i\frac{2\pi}{3}} & -e^{i\frac{4\pi}{3}} & e^{i\frac{4\pi}{3}} \\
                            V_3^{\lor} & 2 & -2 & 0 & e^{i\frac{4\pi}{3}} & -e^{i\frac{4\pi}{3}} & -e^{i\frac{2\pi}{3}} & e^{i\frac{2\pi}{3}} \\
                            V_4 & 1 & 1 & 1 & e^{i\frac{2\pi}{3}} & e^{i\frac{2\pi}{3}} & e^{i\frac{4\pi}{3}} & e^{i\frac{4\pi}{3}} \\
                            V_4^{\lor} & 1 & 1 & 1 & e^{i\frac{4\pi}{3}} & e^{i\frac{4\pi}{3}} & e^{i\frac{2\pi}{3}} & e^{i\frac{2\pi}{3}} \\ \hline
                            W & 2 & -2 & 0 & 1 & -1 & -1 & 1 \\ \hline
                        \end{array}
                    \end{equation*}}
                \caption{Character table of $2\mathcal{T}$.}
            \end{table}

            \begin{table}[H]
                \centering
                {\small
                \begin{equation*}
                        \begin{array}{|c|c|c|c|c|c|c|c|c|}
                            \hline
                            \text{conj. class repr.} & e & B^2 & B & C & C^2 & D & BD & D^3 \\ \hline
                            \text{conj. class order} & 1 & 1 & 6 & 8 & 8 & 6 & 12 & 6\\
                            \hline
                            V_0 & 1 & 1 & 1 & 1 & 1 & 1 & 1 & 1 \\
                            V_1 & 2 & -2 & 0 & 1 & -1 & -\sqrt{2} & 0 & \sqrt{2} \\
                            V_2 & 3 & 3 & -1 & 0 & 0 & 1 & -1 & 1 \\
                            V_3 & 4 & -4 & 0 & -1 & 1 & 0 & 0 & 0 \\
                            V_4 & 3 & 3 & -1 & 0 & 0 & -1 & 1 & -1 \\
                            V_5 & 2 & -2 & 0 & 1 & -1 & \sqrt{2} & 0 & -\sqrt{2} \\
                            V_6 & 1 & 1 & 1 & 1 & 1 & -1 & -1 & -1 \\
                            V_7 & 2 & 2 & 2 & -1 & -1 & 0 & 0 & 0 \\ \hline
                            W & 2 & -2 & 0 & 1 & -1 & -\sqrt{2} & 0 & \sqrt{2} \\ \hline
                        \end{array}
                    \end{equation*}}
                \caption{Character table of $2\mathcal{O}$.}
            \end{table}

            \begin{table}[H]
                \centering
                {\small
                \begin{equation*}
                        \begin{array}{|c|c|c|c|c|c|c|c|c|c|}
                            \hline
                            \text{conj. class repr.} & e & E^2 & E & F & F^2 & EF & (EF)^2 & (EF)^3 & (EF)^4 \\ \hline
                            \text{conj. class order} & 1 & 1 & 30 & 20 & 20 & 12 & 12 & 12 & 12\\
                            \hline
                            V_0 & 1 & 1 & 1 & 1 & 1 & 1 & 1 & 1 & 1 \\
                            V_1 & 2 & -2 & 0 & 1 & -1 & \vp^+ & -\vp^- & \vp^- & -\vp^+ \\
                            V_2 & 3 & 3 & -1 & 0 & 0 & \vp^+ & \vp^- & \vp^- & \vp^+ \\
                            V_3 & 4 & -4 & 0 & -1 & 1 & 1 & -1 & 1 & -1 \\
                            V_4 & 5 & 5 & 1 & -1 & -1 & 0 & 0 & 0 & 0 \\
                            V_5 & 6 & -6 & 0 & 0 & 0 & -1 & 1 & -1 & 1 \\
                            V_6 & 4 & 4 & 0 & 1 & 1 & -1 & -1 & -1 & -1 \\
                            V_7 & 2 & -2 & 0 & 1 & -1 & \vp^- & -\vp^+ & \vp^+ & -\vp^- \\
                            V_8 & 3 & 3 & -1 & 0 & 0 & \vp^- & \vp^+ & \vp^+ & \vp^- \\ \hline
                            W & 2 & -2 & 0 & 1 & -1 & \vp^+ & -\vp^- & \vp^- & -\vp^+ \\ \hline
                        \end{array}
                    \end{equation*}}
                \caption{Character table of $2\mathcal{I}$, with $\vp^\pm\equiv(1\pm\sqrt{5})/2$.}
            \end{table}


    \subsection{Finite subgroups of $\SU(3)$}

        The finite subgroups of $\SU(3)$ are
        \begin{itemize}
            \item the finite subgroups of $\SU(2)$
            \item $\Delta(3n^2)=(\Z_n\times\Z_n)\rtimes\Z_3$ and $\Delta(3n^2)=(\Z_n\times\Z_n)\rtimes S^3$
            \item the exceptional groups
        \end{itemize}
        so there are $2$ infinite series and $5$ exceptional subgroups. Note that they are all divisible by $3$ because the center of $\SU(3)$ is $\Z_3$.

        \begin{theorem*}
            Every abelian finite subgroup of $\SU(3)$ is isomorphic to $\Z_m\times\Z_n$.
        \end{theorem*}

\section{Spacetime geometry: ALE space and orbifolds}\label{app:spacetimegeom}

    Asymptotically locally euclidean (ALE) spaces are a particularly interresting choice of string background to probe with branes for mainly four reasons
    \begin{enumerate}[label=(\roman*)]
        \item they are the resolution (blow-ups) of orbifolds
        \item there are completely classified: they fall in the ADE classification
        \item they only break half of the supersymmetry
        \item they are non-compact therefore we can study them for self-dual type II theory. \todo{Why is that ?}
    \end{enumerate}
    Mathematically, an ALE space is complete riemannian $n$-manifold $M$ such that there exists a compact set $K\subset M$ such that $M\backslash K$ is diffeomorphic to $(\R^n\backslash B_0(R))/G$, where $R\in\R^+_0$ is a radius and $G\subset\O(n)$ a subgroup. Additionally, it is asked that the pulled back metric on $\R^n\backslash B_0(R)$ tends to the euclidean flat metric at infinity.

    If one considers string theory an the orbifold $\R^4/\Gamma$ where $\Gamma$ is a finite sub group of $\SU(2)$, massless states appear from the twisted sector. They are precisely the moduli needed the deform the theory to the one with smooth spacetime, i.e. the resolution of the orbifold. In that sense, is said that the strings know about the metric ALE space and that it is said that strings resolve the singularity. The metric of the ALE space can be recovered if the lagrangian of the resulting field theory is explicitely know, such as for the Wess-Zumino-Witten model. However, it is often not the case.

\section{Graphs}

    The \emph{dimer diagram} of a quiver gauge theory is a graph whose faces represent the gauge groups, the edges represent the bi-fundamental fields and the vertices represent the superpotentials.

\section{Determinantal varieties as transverse spaces}

    \subsection{Basic properties of determinantal varieties}  

            A \emph{determinantal variety} (DV) is a space of matrices with a given upper bound on their ranks. More precisely, given $m,n$ and $r<\min(m,n)$, the DV $Y_r$ of the field $K$ is the set of $m\times n$ matrices over $K$ with rank lower or equal to $r$:
            \begin{equation}
                Y_r\equiv\{M\in M_{m\times n}(K)|\rank M\leq r\}.
            \end{equation}
            Recall that a $k$-minor is the determinant of a $k\times k$ sub-matrix and that the rank of a matrix is equal to the biggest integer such that there is a non-vanishing minor of that size. Imposing $\rank M\leq r$ is therefore equivalent to the vanishing of its $(r+1)\times (r+1)$ minors, as it also implies tha vanishing of the biggest minors. This naturally qualifies $R_r$ as affine varieties embedded in $K^{mn}$. 
            
            Let us denote by $X=(x_{ij})$ an arbitrary $m\times n$ matrix. The independent entries $x_{ij}$ are affine coordinates. The $(r+1)\times(r+1)$ minors are therefore homogeneous polynomials of degree $r+1$. The \emph{determinantal ideal} $I_{r+1}(X)$ is the ideal of $k[X]$ generated by these polynomials. The cooridnate ring is 
            \begin{equation}
                R=k[X]/I_{r+1}(X)
            \end{equation}
            Homogeneity the polynomials implies that $Y_r$ can equivalently be seen as a projective variety in $\bbA^{mn-1}$.

        \subsubsection{Computing the dimension}
            
            Let us compute the dimension of $Y_r$ seen as an affine variety. We consider the space $\bbA^{mn}\times\textbf{Gr}(r,m)$, where $\textbf{Gr}(r,m)$ is the Grassmannian of $r$-planes in an $m$-dimensioanl vector space. Let us define the subsapce
            \begin{equation}
                Z_r\equiv\{(A,W)|Ax\in W\text{ for all } x\in\bbA^{n}\}.
            \end{equation}
            $Y_r$ and $Z_r$ are birationaly equivalent so $\dim Y_r=\dim Z_r$. We want to compute $Z_r$. First we notice that $Z_r$ is a vector bundle over $\textbf{Gr}(r,m)$ and we denote it by $Z_r\xrightarrow[]{\pi_1}\textbf{Gr}(r,m)$. Now, over the Grassmannian $\textbf{Gr}(r,m)$, there is a tautologial vector bundle that we denote by $E_{\textbf{Gr}}\xrightarrow[]{\pi_2}\textbf{Gr}(r,m)$ whose fibers are $\pi^{-1}_2(W)=W\cong\R^r$. Finally, $K^m$ can also be seen as a vector bundle, with fibers $\R^m$. We denote it by $E_{K^n}\xrightarrow[]{\pi_3}K^n$. From $E_{\textbf{Gr}}$ and $E_{K^n}$, we can construct\footnote{Recall that if $E$ and $F$ are vector bundles over $X$, then we can construct a new vetor bundle over $X$, called the Hom-bundle and denoted $\Hom(E,F)$, by defining the fiber over $x\in X$ to be $\Hom(E_x,F_x)$.} the vector bundle $\Hom(E_{\textbf{Gr}},E_{K^n})\xrightarrow[]{\pi_4}\textbf{Gr}(r,m)$. This vector bundle has the same base space and its fibers are $\Hom(\R^m,\R^r)$ which are exactly the same as the ones of $Z_r$. So the two vector bundles are isomorphic:
            \begin{equation}
                Z_r\cong\Hom(E_{\textbf{Gr}},E_{K^n}).
            \end{equation}
            Finally, since the fibers of $\Hom(K^n,E_{\textbf{Gr}})$ have dimension $nr$, we find
            \begin{equation}
                \dim Z_r = \dim\Hom(K^n,E_{\textbf{Gr}}) = \dim\textbf{Gr}(r,m)+nr = r(m-r)+nr.
            \end{equation}
            Finally, we conclude that $Y_r$ is a affine variety of dimension $r(m-r)+nr$.

            \begin{table}[H]
                \centering
                $
                \begin{array}{|c|c|c||c|}
                    \hline
                    m & n & r & \dim_\C Y_r \\ \hline
                    2 & 2 & 1 & 3 \\ \hline
                    3 & 2 & 1 & 4 \\ \hline
                    3 & 3 & 1 & 5 \\ \hline
                    3 & 3 & 2 & 8 \\ \hline
                    4 & 2 & 1 & 5 \\ \hline
                    4 & 3 & 1 & 6 \\ \hline
                    4 & 3 & 2 & 10 \\ \hline
                    4 & 4 & 1 & 7 \\ \hline
                    4 & 4 & 2 & 12 \\ \hline
                    4 & 4 & 3 & 15 \\ \hline
                \end{array}
                $
            \end{table}

        \subsubsection{Singularity}
        
            Determinantal varieties are singular and possess non-commutative resolutions. $Y_r$ is singlar and the singular locus is contained in the subset of matrices with rank strictly lower than $r$. $Z_r$ is a resolution (over the open set of matrices with rank exactly $r$, this map is an isomorphism), it is called the \emph{Springer desingularization} of $\text{Spec}R$.

        \subsubsection{Action and syzygies}

            $Y_r$ naturally acts on $G=\GL(m,K)\times\GL(n,K)$

        \subsection{Young's lattice}

            \emph{Young's lattice} is a lattice $Y$ formed by all integer partitions ordered by inclusion of their Young tableau. It is generally used to to describe the irreducible representation sof the symmetric group\footnote{Two permutations of $S_n$ are equivalent if and only they have they have the same number of cycles of the same sizes. Therefore, the quivalence classes of the symmetric group $S_n$ are parametrized by the partitions of $n$, i.e. by Young diagrams.} $S_n$ together with their branching properties. Conventionnally, Young's lattice is depicted in a Hasse diagram, i.e. with element of the same rank shown at the same height and with links such that the descendance of two elements is the union and the parent is the intersection.

            \begin{figure}[H]
                \centering
                \includegraphics[scale=0.45]{Pictures/youngslattice.png}
                \caption{Young's lattice.}
            \end{figure}

            Young's lattice possess the folling symmetry: the partition $n+n-1+\dots2+1$ of the $n$th triangular number has a Young diagram that looks like a staircase. If we now only keep the elements whose hull is contained in this staircase, we get a subset of Young's lattice. When rank-embedded, this subset clearly has the expected bilateral symmetry of Young's lattice but also a rotational symmetry, which appear more clearly if we move away from this rank-embedding. The rotation group of order $n+1$ acts on this poset\footnote{Partially ordered set.}. Since it has both a bilateral and a rotational symmetry it must also have a dihedral symmetry and, indeed, the dihedral group $\D_{n+1}$ acts faithfully on this set.

            \begin{figure}[H]
                \centering
                \includegraphics[scale=0.45]{Pictures/suter1.png}
                \includegraphics[scale=0.45]{Pictures/suter2.png}
                \caption{Example of dihedral symmetry for $n=4$.}
            \end{figure}

\section{Some derivations}

    \subsection{Invariant configurations for $\C\times\C^2/2\D_n$}\label{app:invconfDn}

        \subsubsection{Gauge field}

            To find the invariant configurations of the gauge group, we use use the bi-index notation and split the sub-blocks of $A_\mu$ in four categories depending on the dimensionality of the representations that they transform in. Note that it is only necessary to check the invariance under the two generators of $2\D_n$ to ensure invariance under the whole group.
            \begin{itemize}
                \item components $A_{\mu;a\alpha_a,b\beta_b}$ are $1\times 1$ blocks that transform as $A_{\mu;a\alpha_a,b\beta_b}\mapsto\sigma_a(\gamma)A_{\mu;a\alpha_a,b\beta_b}\sigma_b(\gamma)^{-1}$. It follows that only the component with $a,b=0,1$ or $a,b=2,3$ can be non-zero to have invariance under $A$. For invariance under $B$, we find that only the component with $a,b=0,2$ or $a,b=1,3$ can be non-zero if $n$ is even and only the component with $a=b$ if $n$ is odd. In conclusion, the invarint configuration under $A$ and $B$ are of the form
                \begin{equation}
                    (A_{\mu;ab})=\begin{bmatrix}
                        \times & 0 & 0 & 0 \\
                        0 & \times & 0 & 0 \\
                        0 & 0 & \times & 0 \\
                        0 & 0 & 0 & \times
                    \end{bmatrix}
                \end{equation}
                regardless of the parity of $n$.
                \item components $A_{\mu;a\alpha_a,r\beta_r}$ are $1\times 2$ blocks that transform as $A_{\mu;a\alpha_a,r\beta_r}\mapsto\sigma_a(\gamma)A_{\mu;a\alpha_a,r\beta_r}\mu_r(\gamma)^{-1}$. More explicitely, each block is of the form $\begin{bmatrix} x_1 & x_2 \end{bmatrix}$ and transforms as
                \begin{equation*}
                    \begin{bmatrix} x_1 & x_2 \end{bmatrix}\mapsto\sigma_a(A)\begin{bmatrix} x_1\zeta^{-r}_{2n} & x_2\zeta^{r}_{2n} \end{bmatrix}
                \end{equation*}
                under the generator $A$. This never invariant unless $a=b=0$. There is no need to check the invariance under $B$ since all these component are already all zero.
                \item components $A_{\mu;r\alpha_r,a\beta_a}$ are $2\times 1$. The situation is exactly the samme as in the previous point: they must all vanish.
                \item components $A_{\mu;r\alpha_r,s\beta_s}$ are $2\times 2$ blocks that transform as $A_{\mu;r\alpha_r,s\beta_s}\mapsto\mu_r(\gamma)A_{\mu;r\alpha_r,s\beta_s}\mu_s(\gamma)^{-1}$. Generically speaking, an invariant block under $A$ must satisfy
                \begin{equation}
                    \begin{bmatrix}
                        \zeta^{r}_{2n} & 0 \\
                        0 & \zeta^{-r}_{2n}
                    \end{bmatrix}
                    \begin{bmatrix}
                        x_1 & x_2 \\
                        x_3 & x_4
                    \end{bmatrix}
                    \begin{bmatrix}
                        \zeta^{-s}_{2n} & 0 \\
                        0 & \zeta^{s}_{2n}
                    \end{bmatrix}=
                    \begin{bmatrix}
                        x_1\zeta^{r-s}_{2n} & x_2\zeta^{r+s}_{2n} \\
                        x_3\zeta^{-r-s}_{2n} & x_4\zeta^{-r+s}_{2n}
                    \end{bmatrix}=
                    \begin{bmatrix}
                        x_1 & x_2 \\
                        x_3 & x_4
                    \end{bmatrix}.
                \end{equation}
                There are two possibilities to have non-vanishing component: $\zeta^{r-s}_{2n}=1$ and $x_2=x_3=0$ or $\zeta^{r+s}_{2n}=1$ and $x_1=x_4=0$ but the latter is actually not possible since $r,s=1,\dots,n-1$. To we find that the blocks must be of the form
                \begin{equation}
                    A_{\mu;r\alpha_r,s\beta_s}=
                        \begin{bmatrix}
                            \times & 0 \\
                            0 & \times
                        \end{bmatrix}
                \end{equation}
                if $r=s$ and vanishing otherwise. For invariance under $B$, a blocks must satisfy
                \begin{equation}
                    \begin{bmatrix}
                        0 & -1 \\
                        1 & 0
                    \end{bmatrix}
                    \begin{bmatrix}
                        x_1 & x_2 \\
                        x_3 & x_4
                    \end{bmatrix}
                    \begin{bmatrix}
                        0 & -1 \\
                        1 & 0
                    \end{bmatrix}=
                    \begin{bmatrix}
                        -x_4 & x_3 \\
                        x_2 & -x_1
                    \end{bmatrix}=
                    \begin{bmatrix}
                        x_1 & x_2 \\
                        x_3 & x_4
                    \end{bmatrix}.
                \end{equation}
                which is only possible if $x_1-x_4$ and $x_2=x_3$. Finally, we find that invariance under $A$ and $B$ imposes the block to be of the form
                \begin{equation}
                    A_{\mu;r\alpha_r,s\beta_s}=
                        \begin{bmatrix}
                            x & 0 \\
                            0 & -x
                        \end{bmatrix}
                \end{equation}
                if $r=s$ and vanishing otherwise.
            \end{itemize}
            The invariant gauge field configurations were found to be of the form
            \begin{equation}
                A_\mu=
                {\tiny
                \left[
                \begin{array}{*{20}c}
                    \times & 0 & 0 & 0 & & & & & & \cdots & 0 \\
                    0 & \times & 0 & 0 & & & & & & & \\
                    0 & 0 & \times & 0 & & & & & & & \\
                    0 & 0 & 0 & \times & & & & & & & \\
                    & & & & x_1 & 0 & & & & & \\
                    & & & & 0 & -x_1 & & & & & \\
                    & & & & & & x_2 & 0 & & & \\
                    & & & & & & 0 & -x_2 & & & \\
                    \vdots & & & & & & & & \ddots & & \vdots \\
                    & & & & & & & & & x_{n-1} & 0 \\
                    0 & & & & & & & & & 0 & -x_{n-1}
            \end{array}
            \right]}\label{eq:invformAmuDn}
            \end{equation}
            where each entry $(i,j)$ is an arbitrary block of size $N_i\times N_j$.

        \subsubsection*{Scalar fields}

            For the real scalar fields $X^m$, we need the action of $2\D_n$ on $\C^3$:
            \begin{equation}
                \begin{bmatrix}
                    z_1\\z_2\\z_3
                \end{bmatrix}\overset{A}{\longmapsto}
                \begin{bmatrix}
                    1 & 0 & 0 \\
                    0 & \zeta_{2n} & 0 \\
                    0 & 0 & \zeta^{-1}_{2n}
                \end{bmatrix}
                \begin{bmatrix}
                    z_1\\z_2\\z_3
                \end{bmatrix},\qquad
                \begin{bmatrix}
                    z_1\\z_2\\z_3
                \end{bmatrix}\overset{B}{\longmapsto}
                \begin{bmatrix}
                    1 & 0 & 0 \\
                    0 & 0 & i \\
                    0 & i & 0
                \end{bmatrix}
                \begin{bmatrix}
                    z_1\\z_2\\z_3
                \end{bmatrix}.\label{eq:RsymDn}
            \end{equation}
            The partitionning of $X^m$ is similar to $A_\mu$. The additional difficulty is come from R-symmetry. Since it acts differently on the different components, we have the study themalmost one by one.
            \begin{itemize}
                \item the fields $X^0$ and $X^1$ are left untouched by R-symmetry, meaning that the invariant configurations have the same form than the gauge field, i.e. \eqref{eq:invformAmuDn}.
                \item $X^{2,3}_{a\alpha_a,b\beta_b}$ transforms under $A$ as $X^{2,3}_{a\alpha_a,b\beta_b}\mapsto \xi_{2n}\sigma_a(A) X^{2,3}_{a\alpha_a,b\beta_b}\sigma_b(A)^{-1}$. The only configurations that are left invariant are therefore the ones such that $ \xi_{2n}\sigma_a(A) \sigma_b(A)^{-1}=1$, which is never the case. So $X^{2,3}_{a\alpha_a,b\beta_b}=0$ for all $a,b=0,\dots,3$.
                \item $X^{2,3}_{a\alpha_a,k\beta_k}$ transforms under $A$ as $X^{2,3}_{a\alpha_a,k\beta_k}\mapsto \xi_{2n}\sigma_a(A) X^{2,3}_{a\alpha_a,k\beta_k}\mu_k(A)^{-1}$. More explicitely, if we denote a block$X^{2,3}_{a\alpha_a,k\beta_k}$ by $\begin{bmatrix} x_1 & x_2 \end{bmatrix}$, we get
                \begin{equation}
                    \begin{bmatrix}
                        \xi^{k+1}_{2n}\sigma_a(A) x_1 & \xi^{-k+1}_{2n}\sigma_a(A) x_2
                    \end{bmatrix}=
                    \begin{bmatrix}
                        x_1 & x_2
                    \end{bmatrix}
                \end{equation}
                therefore we can have $x_1\neq0$ iff $\sigma_a(A)=-1$(i.e. $a=2,3$) and $k=n-1$, and we can have $x_1\neq0$ iff $\sigma_a(A)=1$(i.e. $a=0,1$) and $k=1$.
                \item $X^{2,3}_{k\alpha_k,b\beta_b}$ transforms under $A$ as $X^{2,3}_{k\alpha_k,b\beta_b}\mapsto \xi_{2n}\mu_k(A) X^{2,3}_{k\alpha_k,b\beta_b}\sigma_a(A)^{-1}$. Similarly to the previous case, we can write the blocks $X^{2,3}_{k\alpha_k,a\beta_a}$ as $\begin{bmatrix} x_1 \\ x_2 \end{bmatrix}$, we get
                \begin{equation}
                    \begin{bmatrix}
                        \xi^{k+1}_{2n}\sigma_a(A) x_1 \\ \xi^{-k+1}_{2n}\sigma_a(A) x_2
                    \end{bmatrix}=
                    \begin{bmatrix}
                        x_1 \\ x_2
                    \end{bmatrix}
                \end{equation}
                therefore the conditions are exactly the same: we can have $x_1\neq0$ iff $\sigma_a(A)=-1$(i.e. $a=2,3$) and $k=n-1$, and we can have $x_1\neq0$ iff $\sigma_a(A)=1$(i.e. $a=0,1$) and $k=1$.
                \item $X^{2,3}_{k\alpha_k,l\beta_l}$ transforms under $A$ as $X^{2,3}_{k\alpha_k,l\beta_l}\mapsto \xi_{2n}\mu_k(A) X^{2,3}_{k\alpha_k,l\beta_l}\mu_l(A)^{-1}$. Again, we can write the blocks $X^{2,3}_{k\alpha_k,l\beta_l}$ as $\begin{bmatrix} x_1 & x_2 \\ x_3 & x_4 \end{bmatrix}$ and we get
                \begin{equation}
                    \begin{bmatrix} 
                        \zeta^{k-l+1}_{2n}x_1 & \zeta^{k+l+1}_{2n}x_2 \\
                        \zeta^{-k-l+1}_{2n}x_3 & \zeta^{-k+l+1}_{2n}x_4 
                    \end{bmatrix}=
                    \begin{bmatrix} 
                        x_1 & x_2 \\
                        x_3 & x_4 
                    \end{bmatrix}
                \end{equation}
                therefore, we can have
                \begin{itemize}
                    \item $x_1\neq0$ iff $l=k+1$,
                    \item $x_2\neq0$ iff $l=-k-1$ (not possible),
                    \item $x_3\neq0$ iff $l=-k+1$ (not possible),
                    \item $x_4\neq0$ iff $l=k-1$.
                \end{itemize}
                For $X^{4,5}_{i\alpha_i,j\beta_j}$, the reasonning is exactly the same but with the $R$-symmetry acting as $\zeta^{-1}_{2n}$ instead of $\zeta_{2n}$. After similar computations, we get that the components $X^{4,5}_{a\alpha_a,b\beta_b}$ must be all vanishing too and the components $X^{4,5}_{a\alpha_a,k\beta_k} = \begin{bmatrix} x_1 & x_2 \end{bmatrix}$ can have $x_1\neq0$ iff $a=0,1$ and $k=1$ and $x_2\neq0$ iff $a=2,3$ and $k=n-1$. The same goes for the components $X^{4,5}_{k\alpha_k,b\beta_b}=\begin{bmatrix}x_1\\ x_2\end{bmatrix}$ and, at last, for the components $X^{4,5}_{k\alpha_k,l\beta_l}=\begin{bmatrix} 
                    x_1 & x_2 \\
                    x_3 & x_4 
                \end{bmatrix}$, we find 
                \begin{itemize}
                    \item $x_1\neq0$ iff $l=k-1$,
                    \item $x_2\neq0$ iff $l=-k+1$ (not possible),
                    \item $x_3\neq0$ iff $l=-k-1$ (not possible),
                    \item $x_4\neq0$ iff $l=k+1$.
                \end{itemize}
            \end{itemize}

            We have established what configurations are invariant under the generator $A$, equivalently under the subgroup of $2\D_n$ generated by $A$. What about $B$? The action of $R$-symmetry for $B$ is more tiresome because it is not diagonal, see \eqref{eq:RsymDn}.  This implies that components get exchanged. More precisely, recall our notations $z_1=X^0+iX^1$, etc, if we rewrite \eqref{eq:RsymDn} in terms of real components, we get that
            \begin{equation}
                \begin{bmatrix}
                    X^0\\X^1\\X^2\\X^3\\X^4\\X^5
                \end{bmatrix}\overset{B}{\longmapsto}
                \begin{bmatrix}
                    X^0\\X^1\\-X^5\\X^4\\-X^3\\X^2
                \end{bmatrix}.
            \end{equation}
            For the components $X^2_{a\alpha_a,b\beta_b}$, this implies that $X^2_{a\alpha_a,b\beta_b}=-\sigma_a(B)\sigma_b(B)^{-1}X^5_{a\alpha_a,b\beta_b}$. This completely fixes $X^5_{a\alpha_a,b\beta_b}$ in terms of $X^2_{a\alpha_a,b\beta_b}$. The can be done the other components of $X^2$, they we find that they all determine the ones of $X^5$. Without fully splitting each fields into components, we see that we must have
            \begin{align}
                X^2_{ij}&=-\rho_i(B)X^5_{ij}\rho_j(B)^{-1},\\
                X^3_{ij}&=\rho_i(B)X^4_{ij}\rho_j(B)^{-1},\\
                X^4_{ij}&=-\rho_i(B)X^3_{ij}\rho_j(B)^{-1},\\
                X^5_{ij}&=\rho_i(B)X^2_{ij}\rho_j(B)^{-1},\\
            \end{align}
            bto have invariance under $B$. This equations imply in particular that $X^2_{kl}=-\rho_k(B^2)X^2_{kl}\rho_l(B^2)^{-1}$. Since, $\rho_k(B^2)=-\mathbbm{1}_{2\times 2}$ for every $k$, we get that all components $X^2_{kl}$ must be vanishing. In turn, this implies the components $X^{3,4,5}_{kl}$ must also all vanish. \todo{This cannot be true.}


    \subsection{}\label{app:compsum}

        We want to compute the sum
        \begin{equation}
            \sum^{\lfloor n/3\rfloor}_{a=1}~\left\lfloor \frac{n-3a}{2}+1\right\rfloor = \left\lfloor \frac{n}{3}\right\rfloor + \sum^{\lfloor n/3\rfloor}_{a=1}~\left\lfloor \frac{n-3a}{2}\right\rfloor.
        \end{equation}
        Let us write $n\in\N$ as $n=3m+r$ with $r=0,1$ or $2$ and $m\in\N$. Regardless of $r$, we have $\lfloor n/3\rfloor=m$ and
        \begin{equation}
            \sum^{\lfloor n/3\rfloor}_{a=1}~\left\lfloor \frac{n-3a}{2}\right\rfloor = \sum^{m}_{a=1}~\left\lfloor \frac{3}{2}(m-a)+\frac{r}{2}\right\rfloor = \sum^{m-1}_{a=0}~\left\lfloor \frac{3}{2}a+\frac{r}{2}\right\rfloor.\label{eq:sumfloor}
        \end{equation}
        \begin{itemize}
            \item if $r=0$, then \eqref{eq:sumfloor} becomes
            \begin{equation}
                \sum^{m-1}_{a=0}~\left\lfloor \frac{3}{2}a\right\rfloor = \sum^{m-1}_{a=0}~a+\sum^{m-1}_{a=0}~\left\lfloor \frac{a}{2}\right\rfloor = \frac{(m-1)m}{2}+\sum^{m-1}_{a=0}~\left\lfloor \frac{a}{2}\right\rfloor.
            \end{equation}
            Now if $m$ is even, we have
            \begin{equation}
                \sum^{m-1}_{a=0}~\left\lfloor \frac{a}{2}\right\rfloor = 2\sum^{\left\lfloor \frac{m-1}{2}\right\rfloor}_{a=0}~a = 2\sum^{\frac{m}{2}-1}_{a=0}~a = \left(\frac{m}{2}-1\right)\frac{m}{2}
            \end{equation}
            and if $m$ is odd,
            \begin{equation}
                \sum^{m-1}_{a=0}~\left\lfloor \frac{a}{2}\right\rfloor = 2\sum^{\left\lfloor \frac{m-2}{2}\right\rfloor}_{a=0}~a+\left\lfloor \frac{m-1}{2}\right\rfloor = 2\sum^{ \frac{m-3}{2}}_{a=0}~a+\frac{m-1}{2} = \frac{(m-1)^2}{4}
            \end{equation}
            so
            \begin{equation}
                \sum^{m-1}_{a=0}~\left\lfloor \frac{a}{2}\right\rfloor = 
                \begin{cases}
                    \left(\frac{m}{2}-1\right)\frac{m}{2},\qquad\text{if $m$ is even}\\
                    \frac{(m-1)^2}{4},\qquad\text{if $m$ is odd}
                \end{cases}.\label{eq:suma2floor}
            \end{equation}
            and
            \begin{equation}
                \sum^{m-1}_{a=0}~\left\lfloor \frac{3a}{2}\right\rfloor = 
                \begin{cases}
                    \frac{m(3m-4)}{4},\qquad\text{if $m$ is even}\\
                    \frac{(m-1)(3m-1)}{4},\qquad\text{if $m$ is odd}
                \end{cases}.\label{eq:sum3a2floor}
            \end{equation}
            \item if $r=1$, then \eqref{eq:sumfloor} becomes
            \begin{equation}
                \sum^{m-1}_{a=0}~\left\lfloor \frac{3}{2}a+\frac{1}{2}\right\rfloor = \sum^{m-1}_{a=0}~a+\sum^{m-1}_{a=0}~\left\lfloor \frac{a+1}{2}\right\rfloor = \frac{(m-1)m}{2}+\sum^{m-1}_{a=0}~\left\lfloor \frac{a+1}{2}\right\rfloor
            \end{equation}
            and
            \begin{equation}
                \sum^{m-1}_{a=0}~\left\lfloor \frac{a+1}{2}\right\rfloor = \sum^{m}_{a=1}~\left\lfloor \frac{a}{2}\right\rfloor = \sum^{m}_{a=0}~\left\lfloor \frac{a}{2}\right\rfloor =  
                \begin{cases}
                    \frac{m^2}{4},\qquad\text{if $m$ is even}\\
                    \frac{m^2-1}{4},\qquad\text{if $m$ is odd}
                \end{cases}
            \end{equation}
            by \eqref{eq:suma2floor} so
            \begin{equation}
                \sum^{m-1}_{a=0}~\left\lfloor \frac{3}{2}a+\frac{1}{2}\right\rfloor=
                \begin{cases}
                    \frac{m(3m-2)}{4},\qquad\text{if $m$ is even}\\
                    \frac{3m^2-2m-1}{4},\qquad\text{if $m$ is odd}
                \end{cases}
            \end{equation}
            \item if $r=2$, then \eqref{eq:sumfloor} becomes
            \begin{equation}
                \sum^{m-1}_{a=0}~\left\lfloor \frac{3}{2}a+1\right\rfloor = m+\sum^{m-1}_{a=0}~\left\lfloor \frac{3}{2}a\right\rfloor.
            \end{equation}
            so
            \begin{equation}
                \sum^{m-1}_{a=0}~\left\lfloor \frac{3}{2}a+1\right\rfloor=
                \begin{cases}
                    \frac{3m^2}{4},\qquad\text{if $m$ is even}\\
                    \frac{3m^2+1}{4},\qquad\text{if $m$ is odd}
                \end{cases}
            \end{equation}
            from \eqref{eq:sum3a2floor}.
        \end{itemize}
        Finally, we can write $m=2k$ if $m$ if even and $m=2k+1$ if $m$ is odd in order to distinguish the six different cases. We get
        \begin{align}
            a(n)\equiv\sum^{\lfloor n/3\rfloor}_{a=1}~\left\lfloor \frac{n-3a}{2}+1\right\rfloor&=
            \begin{cases}
                2k+\frac{2k(6k-4)}{4},\qquad\text{if $n=6k$}\\
                2k+\frac{2k(6k-2)}{4},\qquad\text{if $n=6k+1$},\\
                2k+\frac{12k^2}{4},\qquad\text{if $n=6k+2$},\\
                (2k+1)+\frac{2k(6k+2)}{4},\qquad\text{if $n=6k+3$},\\
                (2k+1)+\frac{3(2k+1)^2-2(2k+1)-1}{4},\qquad\text{if $n=6k+4$},\\
                (2k+1)+\frac{3(2k+1)^2+1}{4},\qquad\text{if $n=6k+5$}
            \end{cases}\\
            &=
            \begin{cases}
                3k^2,\qquad\text{if $n=6k$}\\
                3k^2+k,\qquad\text{if $n=6k+1$},\\
                3k^2+2k,\qquad\text{if $n=6k+2$},\\
                3k^2+3k+1,\qquad\text{if $n=6k+3$},\\
                3k^2+4k+1,\qquad\text{if $n=6k+4$},\\
                3k^2+5k+2,\qquad\text{if $n=6k+5$}
            \end{cases}.
        \end{align}
        Starting from $n=1$, the first value of this sequence is : $0,0,1,1,2,3,4,5,7,8,10,12,\dots$. Uppon  further analysis, this correspond to the sequence \href{https://oeis.org/A001399}{\textcolor{blue}{\underline{A001399}}}, that have several interpretations:
        \begin{itemize}
            \item the number of partitions of $n$ into at most 3 parts. This makes sense with our initial problem: finding all the $a,b,c$'s such that $a+b+c=n$,
            \item the number of connected graphs with $3$ nodes and $n$ edges (where multiple edges between the same nodes are allowed),
            \item the number of non-negative solutions to $b+2c+3d=n$,
        \end{itemize}
        as well as many others. Finally, we note that we can simply write
        \begin{equation}
            a(n)=\text{round}\left(\frac{n^2}{12}\right).
        \end{equation}

\pagebreak

\listoftodos

\pagebreak

\printbibliography

\end{document}