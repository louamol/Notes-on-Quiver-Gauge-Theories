%---PACKAGES----------------------------------------
\documentclass[a4paper,11pt]{article}

\usepackage{import}
\import{Packages/}{custom_packages.tex}
\import{Packages/}{custom_macros.tex}

\title{Notes on Quiver Gauge Theories}
\author{Louan Mol\\ \textit{Université Libre de Bruxelles}}

% DOCUMENT -----------------------------

\begin{document}

\maketitle

\vspace{2cm}

\begin{abstract}
    Notes on quiver gauge theories.
\end{abstract}

\tableofcontents

\vfill

Last updated on \today.
  
\pagebreak

\nocite{*}

\section{The supersymmetric sigma model}

    \subsection{Generalities}

        A supersymmetric non-linear sigma model is a theory of maps
        \begin{equation*}
            \Phi:\Sigma^{(D,\mN)}\to\T
        \end{equation*}
        where $\Sigma^{(D|\mN)}$ is the superspace, i.e. a supermanifold\footnote{Recall that a super vector space $V$ is a $\Z_2$-graded vector space. It can always be decomposed in $V=V^0\oplus V^1$, $V^0$ being its even part and $V^1$ its odd part. Denoting by $\R^{m|n}$ the super vector space with even part $\R^m$ and odd part $\R^n$, a super manifold of dimension $m|n$ is a manifold with base space $\R^{m|n}$.} of dimension $D|\mN$ and $\T$ the target space. They are scalar superfields. Let $z=(\xi,\theta)$ be coordinates on the superspace. If $\T$ is a super manifold of dimensions $n|\mN$, we can write $\Phi=(\Phi^1,\dots,\Phi^n)$ and view $\Phi^i$ ($i=1,\dots,n$) as coordinates on $\T$. The action is given by
        \begin{equation}
            S[\Phi]=-\frac{1}{2}\int\d^Dx~D^2(g_{ij}(\Phi)D^a\Phi^iD_a\Phi^j)
        \end{equation}
        where $g$ is a riemannian metric on $\T$, $\alpha=1,\dots,D-1$, $D^2\equiv D^\alpha D_\alpha = \eta^{\alpha\beta}D_\alpha D_\beta$ and
        \begin{equation*}
            D^I_\alpha\equiv 
        \end{equation*}
        with $I=1,\dots,\mN$.

    \subsection{On ALE space}


\pagebreak
\appendix

\section{Notations and conventions}


     

   

\pagebreak

\printbibliography

\end{document}