%---PACKAGES----------------------------------------
\documentclass[a4paper,11pt]{article}

\usepackage{import}
\import{Packages/}{custom_packages.tex}
\import{Packages/}{custom_macros.tex}

\title{\textbf{Notes on Quiver Gauge Theories}}
\author{Louan Mol\\ \textit{Université Libre de Bruxelles}}
\date{}

% DOCUMENT -----------------------------

\begin{document}

\begin{titlepage}
    
    \maketitle

    \thispagestyle{empty}

    \vspace{2cm}

    \begin{abstract}
        Notes on quiver gauge theories.
    \end{abstract}

    \vfill

    \hfill Last updated on \today.
    
\end{titlepage}
  
\pagebreak

\tableofcontents

\pagebreak

\nocite{*}

\section{The brane-world paradigm}

    We consider our world to be a slice in the ten-dimensional spacetime of type II superstring theory, i.e. the worldvolume of a D$3$-brane. More precisely we consider a stack of $n$ $D$3-branes carrying a $\U(n)$ gauge group. The spacetime is therefore not necessarily $\R^{1,9}$ but of the form
    \begin{equation*}
        M = \R^{1,3}\times M^{(6)}.
    \end{equation*}
    This is the so called \emph{brane-world paradigm}. 
    
    Independently from string theory, we can require to have $\mN=1$ supersymmetry in four dimensions. This constrains six-dimensional space of compactification $M^{(6)}$ to be compact, complex, Kähler and to have $\SU(3)$ Holonomy. In other words, $M^{(6)}$ must be a Calabi-Yau threefold. If we let the worldvolume of the D$3$-branes carry the requisite gauge theory while the bulk contains gravity, we can relax the compactness condition an study non-compact threefolds\footnote{Intuitively, this can be understood as a Kaluza-Klein compactification where we take the size of the compact dimensions to infinity. The four-dimensional gravity coupling constant being inversely proportional to this quantity, there is no gravity in this limit.}. In other words, $M^{(6)}$ is an affine variety that localy models a Calabi-Yau threefold. This makes the analysis much simpler and therefore also serves as an argument to ignore gravity on the brane. Thus we have four dimensional- D$3$-branes on which there is a $\U(n)$ gauge group and transverse to which gravity propagates.

    The only smooth Calabi-Yau threefold being $\C^3$, we are lead to consider singular Calabi-Yau manifolds or, more precisely, Calabi-Yau orbifolds that we usually denote $S\equiv M^{(6)}$ String theory being a theory of extended objects, it is well-defined in such singularities. We will see that this singular structure of the geometry will break $\U(n)$ into products of gauge groups.
    
    From the point of view of the orbifold, the D$3$-brane is a point. Consequently, there is a crucial relationship between the D$3$-brane worldvolume theory and the Calabi-Yau singularity: the former parametrizes the latter. In other words, the classical vacuum of the gauge theory should be, in explicit coordinates, the defining equation of $S$.

    Mathematically, this brane-world paradigm is the realization of branes as supports of vector bundles (sheaf). Gauge theories on branes are intimately related to algebraic constructions of stable bundles. In particular, D-brane gauge theories manifest as a natural description of symplectic quotients and their resolutions in geometric invariant theory.

    To summarize in more mathematical terms, our D-brane, together with the stable vector bundle (sheaf) supported thereupon, resolves the transverse Calabi-Yau orbifold which is the vacuum for the gauge theory on the worldvolume as a GIT quotient.

\section{The simplest case : $S=\C^3$}

    \subsection{Generalities}

        Let us consider the simplest non-compact Calabi-Yau threefold: $S=\C^3$. In this case, the spacetime is simply flat space $\R^{1,9}=\R^{1,3}\times\R^6$ with a choice a complex structure on $\R^6$. As mentioned above, the worldvolume theory has a $\U(n)$ gauge group. Type IIB superstring theory is a ten-dimensional $\mN=2$ theory so it has $32$ supercharges. The presence of the breaks the Lorentz symmetry of $\R^{1,9}$ as
        \begin{equation}
            \SO(1,9)\to\SO(1,3)\times\SO(6),
        \end{equation}
        whereby breaking half of the supersymmetries and we are left with $16$ supercharges for the worldvolume theory. In four dimensions, this corresponds to $\mN=4$. We have therefore $\mN=4$ $\U(n)$ SCFT gauge theory on the worldvolume.

        Note that the D$3$-brane will warp the flat space metric to that of $AdS_5\times S^5$ and the bulk geometry is not strictly $\C^3$. However, as stated above, we are only concerned with the local gauge theory and not with gravitational back-reaction, therefore it sufices to consider $S$ as $\C^3$.

    \subsection{Matter content}




\pagebreak
\appendix

\section{Calabi-Yau orbifolds and crepant resolutions}

    Simply put, as \emph{Calabi-Yau manifold} is a Kähler manifold with trivial canonical bundle or, equivalently, with a Kähler metric whose global holonomy is contained in $SU(n)$. A \emph{Calabi-Yau orbifold} is the quotient of a smooth Calabi-Yau manifold by a discrete group action which generically has fixed points. From a geometrical perspective we can try to resolve the orbifold singularity. A resolution $(X,\pi)$ of $\C^n/\Gamma$ is a non-singular complex manifold $X$ of dimension $n$ with a proper biholomorphic map 
    \begin{equation}
        \pi:X\to\C^n/\Gamma
    \end{equation}
    that induces a biholomorphism between dense open sets. 
    \begin{defn}
        A resolution $(X,\pi)$ of $\C^n/\Gamma$ is called a \emph{crepant resolution}\index{resolution!crepant}\footnote{For a resolution of singularities we can define a notion of discrepancy. A crepant resolution is a resolution
        without discrepancy.} if the canonical bundles of $X$ and $\C^n/\Gamma$ are isomorphic, i.e.
        \begin{equation*}
            K_X\cong\pi^*(K_{\C^n/\Gamma}).
        \end{equation*}
    \end{defn}
    Since Calabi-Yau manifolds have trivial canonical bundle, to obtain a Calabi-Yau structure on $X$ one must choose a crepant resolutions of singularities.

    It turns out that the amount of information we know about a crepant resolution of singularities of $\C^n/\Gamma$ depends dramatically on the dimension $n$ of the orbifold. For $n=3$, a crepant resolution always exists but it is not unique; they are related by flops. However all the crepant resolutions have the same Euler and Betti numbers: the \emph{stringy} Betti and Hodge numbers of the orbifold.
   

\pagebreak

\listofmarker
\addcontentsline{toc}{chapter}{\listmarkername}

\pagebreak

\printbibliography

\end{document}