%\begin{titlepage}

	%\begin{tikzpicture}[remember picture,overlay] \node[opacity=0.15,inner sep=0pt] at (current page.center){\includegraphics[width=\paperwidth,height=\paperheight]{Pictures/wave_wall_paper2.png}};
    %\end{tikzpicture}

	%\newpagecolor{gray}\afterpage{\restorepagecolor}
	%\pagecolor{blue!20}

%	\begin{center}
%	\textsc{\LARGE Université Libre de Bruxelles}\\[1.5cm]
	
%	\textsc{\Large Physics Department}\\[0.5cm]
	
%	\textsc{\large Theoretical and Mathematical Physics}\\[0.5cm]
%	\HRule\\[0.8cm]
	
%	{\huge{\bfseries{Notes on Quiver Gauge Theories}}}\\[0.7cm]
	
%	\HRule\\[0.7cm]
	
%	Louan Mol\\
%	\href{mailto:louan.mol@ulb.be}{\texttt{louan.mol@ulb.be}}
	
%	\vspace{3cm}
	
%	{\large\textbf{Abstract}}
%	\end{center}
	
%	    \quad In these notes, we present some basic ideas around the large topic of quiver gauge theories.  The goal is to reproduce and regroup the basics of quiver gauge theories. Note that this is a draft, it may contain a lot of typos, errors and imprecisions. It is only meant as a work support.
	    
%	\vfill

%	\hfill Last update on \today.
	
%\end{titlepage}

\begin{titlepage}

	%\begin{tikzpicture}[remember picture,overlay] \node[opacity=0.15,inner sep=0pt] at (current page.center){\includegraphics[width=\paperwidth,height=\paperheight]{Pictures/wave_wall_paper2.png}};
    %\end{tikzpicture}

	\begin{center}

	{\Huge{\bfseries{Notes on Quiver Gauge Theories}}}\\[0.7cm]

	Louan Mol - \textit{Université Libre de Bruxelles}

	\vspace{10cm}
	
	{\large\textbf{Abstract}}
	\end{center}
	
	    \quad In these notes, we present some basic ideas around the large topic of quiver gauge theories, more precisely about their brane probes construction.  The goal is to reproduce and regroup the basics of these theories depending on the type of singularity of the transverse space (orbifold, toric, del Pezzo, etc). Note that this document is only meant as a work support and it contains a lot of typos, errors and imprecisions.
	    
	\vfill

	\hfill Last update on \today.
	
\end{titlepage}